\documentclass[12pt]{article}
\usepackage[utf8]{inputenc}

\usepackage[T1]{fontenc}
\usepackage{amsmath, amssymb, amsthm}
\usepackage{siunitx}
\usepackage{tikz}
\usepackage{pgfplots}
\usepackage{geometry}
\usepackage{titlesec}
\geometry{a4paper, margin=1in}

\pgfplotsset{compat=1.18}
\usetikzlibrary{3d, arrows.meta}

% Theorem environments
\newtheorem{definition}{Definition}
\newtheorem{proposition}{Proposition}
\newtheorem{remark}{Remark}
\newtheorem{example}{Example}
\newtheorem{theorem}{Theorem}

% Paragraph formatting
\setlength{\parindent}{1.5em}
\setlength{\parskip}{1em}

% Add extra vertical space
\titleformat{\section}{\normalfont\Large\bfseries}{\thesection}{1em}{}
\titlespacing*{\section}{0pt}{2ex plus 1ex minus .2ex}{1ex plus .2ex}

\title{Chapter 03: Numerical series }
\author{Notes from prof zeglaoui course}

\hbadness=10000
\hfuzz=\maxdimen

\newcommand{\N}{\mathbb{N}}
\newcommand{\R}{\mathbb{R}}
\newcommand{\Z}{\mathbb{Z}}
\newcommand{\Q}{\mathbb{Q}}
\newcommand{\C}{\mathbb{C}}
\newcommand{\E}{\mathbb{E}}
\newcommand{\F}{\mathbb{F}}

% --- Derivatives ---
\newcommand{\pd}[2]{\dfrac{\partial #1}{\partial #2}}
\newcommand{\pdd}[3]{\dfrac{\partial^2 #1}{\partial #2\,\partial #3}}
\newcommand{\pddx}[2]{\dfrac{\partial^2 #1}{\partial #2^2}}
\newcommand{\dd}[2]{\dfrac{\mathrm{d} #1}{\mathrm{d} #2}}
\newcommand{\ddd}[2]{\dfrac{\mathrm{d}^2 #1}{\mathrm{d} #2^2}}
\newcommand{\diff}{\,\mathrm{d}}

%% vertical vector macro
\newcommand{\vect}[1]{\begin{pmatrix}#1\end{pmatrix}}


\begin{document}

\maketitle

\vspace{2em}

\section{Generalities}
\begin{definition}
  Let ($U_n$) be a sequence of real( or complex) numbers. By numerical series of the general term $U_n$ , that we denote by $\sum_{n \in \N}U_n$, we mean the couple$((U_n),(\mathcal{U}_n))$  of real (or complex) sequences, where 
  ($\mathcal{U}_n$) is the sequence  of partial sums of $\sum U_n$ , defined by ;
  \[
    \mathcal{U}_n = u_0 + u_1 + \dots + u_n = \sum_{k = 0}^n U_k \quad;  \forall_k \in \N, 
  \]
\end{definition}

\begin{example}
  ..
  
\begin{center}
\includegraphics[width=0.8\textwidth]{Generalities_ex01.jpg}
\end{center}
\end{example}

\begin{definition}
  Let $\sum_{n \in \N} U_n$ numerical series and ($\mathcal{U}_n$). its sequence of partial sums

  1) the series $\sum_{n\in\N} U_n$ is said to be convergent($Cv$)if 

  ($\mathcal{U}_n$ converges; the limit
  $\mathcal{U} = \lim\limits_{n \to +\infty}\mathcal{U}_n$ is called the sum of the series $\sum_{n\in\N}$, often 
  denoted by : $\mathcal{U} = \sum_{n = 0}^{\infty} U_n$

  2) the series $\sum U_n$ is said to be divergent if ($\mathcal{U}_n$) diverges

  3) the nature of $\sum_{n\in \N}U_n$ is the fact that it is convergent or divergent
\end{definition}

\begin{example}
 1) 

\begin{center}
\includegraphics[width=0.8\textwidth]{Generalities_ex02.jpg}
\end{center}

2)

\begin{center}
\includegraphics[width=0.8\textwidth]{Generalities_ex02.jpg}
\end{center}
\end{example}

\begin{theorem}
  \[
    \sum U_n \quad converge \implies \lim\limits_{n \to + \infty} (U_n) = 0
  \]
  \textbf{Proof}

  Because $U_n = \mathcal{U}_n - \mathcal{U}_{n-1} \quad, \forall n$
  
  \[
    U_n  \quad converges \Leftrightarrow (\mathcal{U}_n)converges \implies \lim\limits U_n = \mathcal{U} -\mathcal{U} = 0
  \]
\end{theorem}

\textbf{Remark}

\[
\lim U_n \neq 0 \implies \sum U_n \text{ diverges}
\]

\begin{example}
  the harmonic series

\begin{center}
\includegraphics[width=0.8\textwidth]{Generalities_ex03.jpg}
\end{center}
\end{example}

\section{Positive series:}

\begin{definition}
  the real series  $\sum_{n \in \N} U_n$ is said to be positive if there exists $N \in \N$ st: $\forall n \geq N\quad U_n \geq 0$
\end{definition}

\textbf{Remark:}
if $\sum_{n \in \N} U_n$ is positive then the partial sums sequence $\mathcal{U}_n$ is an increasing sequence


Recall that if  $(\alpha_n)_{n \in \N}$ is an increasing real sequence either ($\alpha_n$) converges to its supremum(i.e $\lim \alpha_n = sup_{n\in \N}(\alpha_n)$ when it's upper bounded) or $\lim \alpha_n = +\infty$ if not

\underline{\textbf{proposition:}} comparision criterion

Let $\sum_{n \in \N} U_n$ and $\sum_{n \in \N} V_n$ be two positive series st:
\[
  \exists N \in \N \quad, \forall n \geq N\quad U_n \leq V_n
\]
so, $\sum_{n \in \N} V_n$ converges $\implies$  $\sum_{n \in \N} U_n$ converges

then  $\sum_{n \in \N} U_n$ diverges $\implies$ $\sum_{n \in \N} V_n$ diverges

\underline{\textbf{Proof:}}

if ($U_n)$ is the sequence of partial sums of $\sum_{n \in \N} U_n$

and ($V_n)$ is the sequence of partial sums of $\sum_{n \in \N} V_n$

$U_n \leq V_n$ hence:
$\sum_{n \in \N} V_n$ converges(converges) $\Leftrightarrow$$(V_n)_n$converges $\implies$ $(V_n)_n$ is upper bounded 
$\implies$ $(U_n)_n$ is upper bounded too $\implies(U_n)$ converges $\implies$$\sum_{n \in \N} U_n$ converges

\begin{example}
  ..

\begin{center}
\includegraphics[width=0.8\textwidth]{positive_series_ex01.jpg}
\end{center}
\end{example}


\underline{\textbf{Corollary:}} equivalance criterion

Let $\sum_{n \in \N} U_n$ and  $\sum_{n \in \N} V_n$ be a positive  series

1)
if $\exists a > b >0, \quad \exists N \in \N;\quad \forall n> N$

$0 \leq a \leq \dfrac{U_n}{V_n}\leq b $

$U_n = O(V_n)$

then  $\sum_{n \in \N} U_n$ and $\sum_{n \in \N} V_n$ have the same nature

2) if $\lim\dfrac{U_n}{V_n} = l > 0 $ then $\sum_{n \in \N} U_n$ and $\sum_{n \in \N} U_n$ have the same nature also($U_n \sim V_n$)

3) if $U_n = O(V_n)$, that is   $\lim\dfrac{U_n}{V_n} = 0$ then :

$\sum V_n$ converges $\implies \sum U_n$ converges (also ,$\sum U_n$ div $\implies \sum V_n$ div) 

\underline{\textbf{proof:}}
\begin{center}
\includegraphics[width=0.8\textwidth]{proof_Landou.jpg}
\end{center}


\underline{\textbf{Landau's symbols:}}
\[U_n = o(V_n) \text{ if } \lim\dfrac{U_n}{V_n} = 0\]
\[U_n \sim (V_n) \text{ if } \lim\dfrac{U_n}{V_n} = 1\]
\[U_n = O(V_n) \text{ if } (\dfrac{U_n}{V_n}) \text{ is bounded}\]

\begin{example}
  for $\alpha > 0$
  \[
    sin(\dfrac{1}{n^\alpha}) \sim \dfrac{1}{n^\alpha}
  \]
  \[
    ln(1 +\dfrac{1}{n^\alpha} ) \sim \dfrac{1}{n^\alpha}
  \]
\end{example}

\underline{\textbf{Proposition}} D'Alenbert's and Cauchy's criterion

1)(D'Alenbert) If:
If $\lim \dfrac{U_{n+1}}{U_n} = l \geq 0$ , then $\sum U_n$ converges if $0 \leq l < 1$   

$\quad \sum U_n$ div if $  l > 1$ 

2)(Cauchy) If:

$\lim\sqrt[n]{U_n} = l \geq 0\quad,$ then  $\sum U_n$ converges if $0 \leq l <1$   

$\quad \sum U_n$ div if $  l > 1$ 

3) if $l =1 $ , we may use another criterion

\begin{example}
..
\begin{center}
\includegraphics[width=0.8\textwidth]{Alber_Cauchy_ex01.jpg}
\end{center}
\end{example}

\underline{\textbf{Remark + example}}


\begin{center}
\includegraphics[width=0.8\textwidth]{Remark_ex.jpg}
\end{center}

\begin{theorem} Comparision with an upper umpropre integral
  Let $f: \left[0, +\infty \right[ \to \left[0, +\infty \right[$
      be a decreasing continuous function and $U_n = f(n)$

      the series $\sum_{n \in \N} U_n = \sum_{n \in \N} f(n)$ have the same nature as the sequence($\int_0^n f(x)dx)_{n\in \N}$ Furthermore:
\[
  \forall p \in \N \quad \int_{p + 1}^{\infty} f(x)dx \leq R_p = U -U_p \leq \int_{p}^{\infty} f(x)dx
\]
\end{theorem}

\underline{\textbf{Proof}}
Since $f$ is decreasing on $\left[h, h+1\right]$, then:
\[ f(h+1) \leq f(x) \leq f(h), \quad \forall x \in \left[h, h+1\right] \]
\[ U_{h+1} \leq f(x) \leq U_n\]
\[ U_{h+1} \int_h^{h+1} dx \leq \int_h^{h+1}f(x)dx \leq U_n\int_h^{h+1} dx\]
\[ U_{h+1}  \leq \int_h^{h+1}f(x)dx \leq U_n\]
\[\implies \sum_{h = 0}^{n-1} U_{h+1}  \leq \sum_{h = 0}^{n-1}\int_h^{h+1}f(x)dx \leq \sum_{h = 0}^{n-1}U_n\]

\[ U_n - U_0\leq  \int_0^n f(x) dx \leq U_{n-1}\]
($U_n)$ and  ($\int_0^n f(x) dx$) are increasing so:
\[
\begin{aligned}
  \sum U_n \text{ converges } &\Leftrightarrow (U_n) \text{ converges } \Leftrightarrow (U_{n-1}) \text{converges} \\
                       &\Leftrightarrow (U_n) \text{ is upperbounded} \\
                       &\implies (\int_0^n f(x) dx) \text{ is upperbounded}\\
                       &\implies (\int_0^n f(x) dx) \text{ converges }\\
                       \text{ conversly:} \int_0^n f(x) dx \text{ converges } \\
                       &\Leftrightarrow  \int_0^n f(x) dx \text{ is upperbouned} \\
                       &\implies (U_n)  \text{ is upperbounded} \\
                       &\implies (U_n) \text { converges }
\end{aligned}
\]

\begin{example}
  ..
\begin{center}
\includegraphics[width=0.8\textwidth]{Bertrand_series.jpg}
\end{center}
\end{example}

\section{Alternating series}

\begin{definition}
  A numerical series $\sum U_n$  is said to be alternating  if its general term $U_n$ change sign infinitly many times 
\end{definition}

\underline{\textbf{Remark}}


$\sum U_n$ is alternating if there exist two subsequences 

$(a_n) = (U_{\varphi_n}) \text{ and } (b_n) = (U_{\psi_n}) $st:

$(a_n) > 0$ and $ (b_n) < 0 $

\begin{example}
\begin{center}
\includegraphics[width=0.8\textwidth]{Alternating_ex01.jpg}
\end{center}
\end{example}

\begin{definition}
  Absolute convergence and semi convergence

  A numerical series $\sum U_n$ is said to be absolutely convergent if the positive series $\sum_{n \in \N} |U_n|$ converges
\end{definition}

\begin{theorem}
  if $\sum U_n$ is absolutely convergent then $\sum U_n$ converges 
\end{theorem}
\underline{\textbf{Proof}}

Let $V_n = U_n + |U_n| > 0 \quad, \forall n\in \N  $  and $0 \leq V_n \leq 2|U_n|$ , by comparision criterion

if $\sum U_n$ converges then $\sum V_n$ converges $\implies \sum U_n = \sum V_n -\sum |U_n|$ converges

\begin{example}
  The Abel's series $\sum \dfrac{cos(n\theta)}{n^\alpha}$ and  $\sum \dfrac{sin(n\theta)}{n^\alpha}$ convergent absolutely 

  if  $ \alpha >1 $ because $\left|\dfrac{cos(n\theta)}{n^\alpha}\right| \leq \dfrac{1}{n^\alpha}$ and $\left|\dfrac{sin(n\theta)}{n^\alpha}\right| \leq \dfrac{1}{n^\alpha}$
\end{example}

\begin{definition}
  We say that $\sum U_n$  conditionally (or semi) convergent, if $\sum U_n$ converges and $\sum |U_n|$ divergent
\end{definition}

\begin{example}
  ..
\begin{center}
\includegraphics[width=0.8\textwidth]{semi_con_ex_part01.jpg}
\end{center}

\begin{center}
\includegraphics[width=0.8\textwidth]{semi_con_ex_part02.jpg}
\end{center}
\end{example}

\begin{theorem}\underline{Leibniz}

  Let $U_n = (-1)^n a_n \mid a_n > 0 \quad,a_{n+1} \leq a_n \quad,\forall n \in \N$ and $\lim a_n = 0$ ,then:

  $\sum (-1)^na_n$ Converges
\end{theorem}
\begin{example}
  $\sum \dfrac{(-1)^n}{n^\alpha}$ converges $\Leftrightarrow \alpha >0$

\[S = \sum_{n = 3}^{+\infty}\frac{1}{n^2 + 3n +2 } = \sum_{n = 3}^{+\infty}\frac{1}{(n+1)(n+2)} = \sum_{n = 3}^{+\infty}\left(\frac{1}{n+1} -\frac{1}{n+2}\right) = 1/4\]
\end{example}

\begin{theorem}(Abel's criterion)
  If: $U_n = a_nb_n$ st:

  1) $a_n \geq 0\quad; a_{n+1} \leq a_n \quad \forall n \in \N \text{ and } \lim a_n = 0$

  2)$\exists M \geq 0 \quad, \forall n \in \N \quad \left|\sum_{h = 0}^n b_h\right| \leq M \text{ then } \sum U_n \text{ converges}$
\end{theorem}

\begin{example}

  .

\begin{center}
\includegraphics[width=0.8\textwidth]{Abel_ex_part01.jpg}
\end{center}

\begin{center}
\includegraphics[width=0.8\textwidth]{Abel_ex_part02.jpg}
\end{center}
\end{example}


\end{document}


