\documentclass{article}
\usepackage[utf8]{inputenc}
\usepackage{amsmath}
\usepackage{graphicx}

\title{Hello World}
\author{someone}
\date{date}

\begin{document}
\maketitle

\section*{Introduction}
\begin{enumerate}
  \item let's try a \textbf{formula} $e^{i\pi} + 1 = 0$

\begin{equation}
\label{limit}
e=\lim_{n\to\infty} \left(1+ \frac{1}{n}\right)^n = \lim_{n\to\infty} \frac{n}{\sqrt[n]{n!}}
\end{equation}

\item but we can do \textit{another}:

\begin{equation}
  \label{infinite_sum}
e= \sum_{n = 0}^{\infty} \frac{1}{n!}
\end{equation}

\item We can also use a \underline{\textbf{continued fraction}}

$$
e = 2+\frac{1}{1 + \frac{1}{2 + \frac{2}{3+ \frac{3}{4 + \dots}}}} % there is ddots for digonal dots
$$

\end{enumerate}

\section*{More Formulas}
$$
\int_a^bf(x)dx
$$

$$\iiint f(x,y,z)dxdydz$$

$$\vec{v} = <{v_1 , v_2, v_3}> $$
$$\vec{v}\cdot \vec{w}$$

$$
\begin{bmatrix}
1 & 2 & 3 \\
4 & 5 & 6 \\
7 & 8 & 9 \\
\end{bmatrix}
$$ 
equation \ref{infinite_sum} was very cool\\
equation \ref{limit} was very cool
\end{document}
