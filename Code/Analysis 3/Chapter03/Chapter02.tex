\documentclass[12pt]{article}
\usepackage[utf8]{inputenc}
\usepackage[T1]{fontenc}
\usepackage{amsmath, amssymb, amsthm}
\usepackage{siunitx}
\usepackage{tikz}
\usepackage{pgfplots}
\usepackage{geometry}
\usepackage{titlesec}
\geometry{a4paper, margin=1in}

\pgfplotsset{compat=1.18}
\usetikzlibrary{3d, arrows.meta}

% Theorem environments
\newtheorem{definition}{Definition}
\newtheorem{proposition}{Proposition}
\newtheorem{remark}{Remark}
\newtheorem{example}{Example}
\newtheorem{theorem}{Theorem}

% Paragraph formatting
\setlength{\parindent}{1.5em}
\setlength{\parskip}{1em}

% Add extra vertical space
\titleformat{\section}{\normalfont\Large\bfseries}{\thesection}{1em}{}
\titlespacing*{\section}{0pt}{2ex plus 1ex minus .2ex}{1ex plus .2ex}

\title{Chapter 02: Vector Analysis }
\author{Notes from prof zeglaoui course}

\hbadness=10000
\hfuzz=\maxdimen

\newcommand{\N}{\mathbb{N}}
\newcommand{\R}{\mathbb{R}}
\newcommand{\Z}{\mathbb{Z}}
\newcommand{\Q}{\mathbb{Q}}
\newcommand{\C}{\mathbb{C}}
\newcommand{\E}{\mathbb{E}}
\newcommand{\F}{\mathbb{F}}

% --- Derivatives ---
\newcommand{\pd}[2]{\dfrac{\partial #1}{\partial #2}}
\newcommand{\pdd}[3]{\dfrac{\partial^2 #1}{\partial #2\,\partial #3}}
\newcommand{\pddx}[2]{\dfrac{\partial^2 #1}{\partial #2^2}}
\newcommand{\dd}[2]{\dfrac{\mathrm{d} #1}{\mathrm{d} #2}}
\newcommand{\ddd}[2]{\dfrac{\mathrm{d}^2 #1}{\mathrm{d} #2^2}}
\newcommand{\diff}{\,\mathrm{d}}

%% vertical vector macro
\newcommand{\vect}[1]{\begin{pmatrix}#1\end{pmatrix}}


\begin{document}

\maketitle

\vspace{2em}
\section{vector fields and scalar fields}
\begin{definition}
  1) We call a scalar field defined on a region $D$  $\subset\R^2$ (respectively on $\Omega \subset \R^3$), a real function defined on $D$ (respectively $\Omega$)

2) a vector field F over $D \subset \R^2$ (resp $\Omega )$ is a function F that assigns to any $(x,y) \in D$ a vector 
$F(x,y) = M(x,y)\vec{i} + N(x,y)\vec{j} \quad \text{ where } M,N \in \mathcal F(D, \R)$

3) a vector field F over $\Omega \subset \R^3$ is a function F that assignes to any $(x,y,z) \in \Omega$ a vector 
$F (x,y,z) = P(x,y,z)\vec{i} + Q(x,y,z)\vec{j} + R(x,y,z)\vec{k}$
\end{definition}


\begin{example}
  1) the position vector field 
\begin{center}
\includegraphics[width=0.8\textwidth]{2.1example01.jpg}
\end{center}
\end{example}

\begin{definition}
  A vector field $F$ is conservative if there exists a

  potential function $f ,\quad F = grad(f)$ 
\end{definition} 


\begin{example}
...
\begin{center}
\includegraphics[width=0.8\textwidth]{2.1.example02.jpg}
\end{center}

\end{example}

\begin{theorem}
  A vector field $F= M\vec{i} + N\vec{j}$ over a convex region $D \subset \R^2$ is conservative if and only if 
  \[ \pd{M}{y} = \pd{N}{x} \]
\end{theorem}

\begin{proof}
  if $F$ is conservative
  \[ F = \pd{f}{x}\vec{i} + \pd{f}{y}\vec{j} \quad \text{ where $f$ is of class  $C^2$}\]
 $$ 
    \pd{M}{y} - \pd{N}{x} = \pd{}{y}\left(\pd{f}{x}\right) -\pd{}{x}\left(\pd{f}{y}\right) =
    \pdd{f}{y}{x} - \pdd{f}{x}{y} = 0
 $$ 
  Shwarz' theorem

  $\Leftrightarrow $
  \[
 f(x,y) = \int_0^1 \left[(x-x_0)M(x_0+t(x-x_0),y_0 +t(y-y_0))+ (y-y_0)N(x_0+t(x-x_0),y_0 +t(y-y_0)) \right]dt
  \]
\begin{center}
\includegraphics[width=0.8\textwidth]{2.1.proof.jpg}
\end{center}
\end{proof}

\begin{example}
  .
\begin{center}
\includegraphics[width=0.8\textwidth]{2.1.example03.jpg}
\end{center}
\end{example}

\begin{definition}
  Let  $ F:(x,y,z) \to P(x,y,z)\vec{i}+Q(x,y,z) \vec{j}+ R(x,y,z)\vec{z}$
   be a vector field over $\Omega \subset \R^3$. Such that the first partial deriatives of $F$ exists

   the curl of $F$, denoted by $Curl(F)$ (Or $\nabla\times F$)
   is the vector fiels over $\Omega $ defind by :
   \[
     \begin{aligned}
     Curl(F) =  \vect{
       \vec{i} &\quad \vec{j} &\quad \vec{k} \\
       \pd{}{x} &\quad \pd{}{y} &\quad \pd{}{z} \\
       P &\quad Q &\quad R
     }
   \end{aligned} = (\pd{R}{y}-\pd{Q}{z})\vec{i} +(\pd{P}{z}- \pd{R}{x})\vec{j} + (\pd{Q}{x}-\pd{P}{y})\vec{k}
   \]
   We say that $F$ is irrotationnal vector field if  $Curl(F) = 0$
\end{definition}
\begin{example}
.
\begin{center}
\includegraphics[width=0.8\textwidth]{2.1.example04.jpg}
\end{center}
\end{example}
\begin{theorem}
  Let $F = P\vec{i} + Q\vec{j} + R\vec{k}$ be vector field over a convex domain $\Omega \subset \R^3$ of class $C^1$
  then $F$ is conservative $\Leftrightarrow Curl(F) = 0$ 
\end{theorem}

\textbf{proof}
\begin{center}
\includegraphics[width=0.8\textwidth]{2.1.proof2.1.jpg}
\end{center}

\begin{center}
\includegraphics[width=0.8\textwidth]{2.1.proof2.2.jpg}
\end{center}

\begin{example}
.
\begin{center}
\includegraphics[width=0.8\textwidth]{2.1.example05.jpg}
\end{center}
\end{example}

\begin{definition}
  Divergence of vector field-Laplacian of scalar field

  1)$div(F)$= $\nabla.F$  $\to$
  $
    \begin{cases}
      F = M\vec{i} + N\vec{j} \implies div(F) = \pd{M}{x} + \pd{N}{y} \\
      F = P\vec{i} + Q\vec{j} + R\vec{k}\implies div(F) = \pd{P}{x} + \pd{Q}{y} + \pd{R}{z} \\
    \end{cases}
  $

  2) If $F = grad(f) \implies div(F) = div(grad(f)) = \Delta(f) \quad \text{ the Laplacian of f}$

  3) F is called divergence free if $div(F) = 0$

  if is called harmonic function if $\Delta(f) = 0  \to$ 
  $
  \begin{cases}
    \pddx{f}{x} + \pddx{f}{y}  =0  \\
    \pddx{f}{x} + \pddx{f}{y} + \pddx{f}{z} = 0  
  \end{cases}
  $
\end{definition}

\textbf{Remark:} If $F$ is of class $C^2$ vector field, then $div(curl(F)) = 0$, using Schwarz's theorem

\section{Line integrals}
\subsection{Line integral of scalar field}

Let $f : D \to \R$ a continuous function over $D \subset \R^2(resp: D\subset \R^3)$, let 
\[\Gamma :
\begin{cases}
  \left[a,b\right] \to D \\
    t \mapsto \gamma(t)
    \end{cases}
  \]
of $C^1$ curve on $D$

the line integral of $f$ over $\Gamma$ denoted by
$\quad \int_\Gamma f dl$ is a real number

\[\int_\Gamma fdl = \int_a^b f(\gamma(t)) ||\gamma'(t)|| dt \quad \text{ where} ||\gamma'(t)||= \sqrt{(x'(t))^2 +(y'(t))^2+(z'(t))^2}\]

in particular if $f \equiv 1$ ;

$\int_\Gamma dl = l(\Gamma) \quad \text{length of $\Gamma$}$
$= \int_a^b ||\gamma'(t)||dt$

$\int_\Gamma f dl$ is independant to parametrization $(\left[a,b\right], \gamma) \text{ of }\Gamma$

2) If $\Gamma$ is piece-wise $C^1$ ; i.e $\exists k \geq 2$
\[ \forall i\in\{1,\dots,k\} ; \Gamma_i \text{ is }C^1\quad; \Gamma_i \cap \Gamma_{i+1} = \{pt\} \quad \text{ and } \Gamma_i \cap \Gamma_j = \phi \quad if \quad |i-j| \geq 2 \]
then:
\[\int_\Gamma fdl = \sum_{i=1}^k \int_{\Gamma_i}fdl\]

\begin{example}
  .
\begin{center}
\includegraphics[width=0.8\textwidth]{2.1.example06.jpg}
\end{center}
\end{example}

\subsection{Line integral of a vector field}

Let $\Gamma$ a curve parametrized by $\left[a,b\right] \xrightarrow{\gamma} \gamma(t)$ and F a continuous vector field

the line integral of F over $\Gamma$ ( or the circulation of F to $\Gamma$) is :
\[\int_\Gamma F.dr = \int_a^b \left[F(\gamma(t))\cdot\gamma'(t)\right]dt\]

which is independant of the choice of parametrization

in $\R^2$ if $\gamma(t) = (x (t),y(t)) $ and $F = M\vec{i} + N\vec{j}$

\[\int_\Gamma F.dr = \int_a^b \left[x'(t)M(x(t),y(t)) + y'(t) N (x(t),y(t))\right]dt\]

In $\R^3$ if $\gamma(t) = (x(t),y(t),z(t))$ and $F = P\vec{i} + Q\vec{j} + R\vec{k}$ 

then:
\[\int_\Gamma F.dr = \int_a^b \left[x'(t)P(x(t),y(t),z(t)) + y'(t)Q(x(t),y(t),z(t))+ z'(t)R(x(t),y(t),z(t))\right] dt\]

\underline{\textbf{Remark:}}

We denoted by $\int_\Gamma Mdx + N dy = \int_\Gamma F.dr$  int $\R^2$ and  by

$\int_\Gamma Pdx + Qdy + Rdz = \int_\Gamma F.dr $ in $\R^3$

this is the differential representation of the integral of F along to $\Gamma$


\underline{\textbf{Remark:}}
If $F = grad(f)$ is conservative vector field then:
\[
\begin{aligned}
\begin{cases}
  df = \pd{f}{x}dx + \pd{f}{y}dy = F.dr \quad \mid dr = dx\vec{i} + dy\vec{j} \\
  df = \pd{f}{x}dx + \pd{f}{y}dy + \pd{f}{z} = F.dr \quad \mid dr = dx\vec{i} + dy\vec{j} + dz\vec{k} \\
\end{cases}
\end{aligned}
\]

\begin{example}
 .. 
\begin{center}
\includegraphics[width=0.8\textwidth]{Line_int_ex02.jpg}
\end{center}
\end{example}


\begin{example}
 .. 
\begin{center}
\includegraphics[width=0.8\textwidth]{Line_int_ex03.jpg}
\end{center}
\end{example}

\begin{theorem}
  Let $F = grad(f)$ be a conservative vector field over $D : \Gamma \subset D$, parametrized by ($\left[a,b\right]$,$\gamma$), then:
  \[
    \int_\Gamma F.dr = f(\gamma(b)) - f(\gamma(a))
  \]

  in particular, if $\Gamma$ is a closed path  ($\gamma(a) = \gamma(b)$) then : $\oint F.dr = 0$

  \textbf{Proof}
  \[
    F = grad(f) = \pd{f}{x}\vec{i} + \pd{f}{y}\vec{j}; 
    \begin{cases}
      \gamma(t) = x(t)\vec{i} + y(t)\vec{j} \\
      \gamma'(t) = x'(t)\vec{i} + y'(t)\vec{j} \\
    \end{cases}
  \]

  \[
    F(\gamma(t)).\gamma'(t) = x'(t)\pd{f}{x}(x(t),y(t)) + y'(t) \pd{f}{y}(x(t),y(t)) =(f \circ \gamma)'t \quad \mid (f \circ \gamma)(t) = f(x(t),y(t))
  \]
  \[
    \int_\Gamma F.dr = \int_\Gamma grad(f).dr = \int_a^b (f\circ\gamma)'(t) dt =\left[f(\gamma(t))\right]_a^b
  \]
  \[
    = f(\gamma(b)) -f(\gamma(a))
  \]
\end{theorem}

\begin{theorem}
  (Green's theorem)
  
  Let D be a simply connected and connected region  of $\R^2$ with piece-wise $C^1 $ boundary simple curve 
  $\Gamma = \partial D$

  and $F = M\vec{i}  + N\vec{j}$ be a $C^1$vector field over $D$, Then :
  \[\oint_\Gamma F.dr = \oint_\Gamma Mdx + Ndy = \iint_D (\pd{N}{x}- \pd{M}{y})(x,y)dxdy\]
\end{theorem}

\begin{example}
  ..
\begin{center}
\includegraphics[width=0.8\textwidth]{Line_int_ex04.jpg}
\end{center}
\end{example}

\begin{theorem}
  \underline{Corollary}
  \underline{Line Integral area}

  if $D$ is a plane region bounded by a piece-wise simple closed curve $\Gamma = \partial D$ oriented counterclockwise, then $A(D)= \frac{1}{2} \oint_{\Omega = \partial D} (xdy -ydx)$

  \textbf{Proof}

  \[ M(x,y) = -\frac{1}{2}y; \quad N(x,y) = \frac{1}{2}x\]

  by Green's theorem;
  \[\oint_{\partial D} Mdx + Ndy = \iint_D (\pd{N}{x}- \pd{M}{y})(x,y)dxdy\]

\end{theorem}

\begin{example}
.
\begin{center}
\includegraphics[width=0.8\textwidth]{Line_int_ex01.jpg}
\end{center}
\end{example}

\section{Surface integral}
\subsection{Parametric surface}

\begin{definition}
  Let x,y,z be a three functions of u and v, that are continuous on $D \subset \R^2$. The set $S$ of  points (x,y,z) given by :  $r(u,v) = x(u,v)\vec{i}+y(u,v)\vec{j}+z(u,v)\vec{k}$
  be a parametrized surface. The following equations:
  \[S:
    \begin{cases}
      x= x(u,v) \\
      y= y(u,v)\quad, \text{are called the parametric equation of S} \\
      z= z(u,v) 
    \end{cases}
  \]
\end{definition}
\begin{example}
..
\begin{center}
\includegraphics[width=0.8\textwidth]{Parametric_Surface_ex01.jpg}
\end{center}

..

\begin{center}
\includegraphics[width=0.8\textwidth]{Parametric_Surface_ex02.jpg}
\end{center}
\end{example}

\begin{definition}
  Let $(D,r)$ be a parametrization of the surface S.

  We say that S is smooth (or $C^1$) if x,y,z are $C^1$ or the vector field $V$  is $C^1$ on D

  we say that S is piece-wise smooth. if 
  $\exists k \in \N^{*}; S =S_1 \cup S_2\cup ... \cup S_k \mid S_i \cup S_j$ is at most a curve and $S_i$ is smooth

  Let S be smooth surface. The tangent plane of S at a point $(x_0,y_0,z_0) \in S$ is the affine space
  \[T_{(x_0,y_0,z_0)}S : 
  {
  \begin{cases}
    x = x_0 + (u-u_0) \pd{x}{u}(u_0,v_0)+ (v-v_0) \pd{x}{v}(u_0,v_0)\\
    y = y_0 + (u-u_0) \pd{y}{u}(u_0,v_0)+ (v-v_0) \pd{y}{v}(u_0,v_0) \\
    z = x_0 + (u-u_0) \pd{z}{u}(u_0,v_0)+ (v-v_0) \pd{z}{v}(u_0,v_0)
  \end{cases}
  }\]
  \[ :r:(u,v) \mapsto r(u_0,v_0) + (u -u_0)\pd{r}{u}(u_0,v_0) + (v-v_0)\pd{r}{v}(u_0,v_0)\]
  \[ r(u_0,v_0) = x_0\vec{i} + y_0\vec{j} + z_0\vec{k}\quad;\quad \pd{r}{u}= r_u \quad ; \quad \pd{r}{v}= r_v\]

  the normal vector field of S ; denoted by N is :

  $N = r_u\times r_v = \pd{r}{u} \times \pd{r}{v}$

  \[
  \begin{aligned}
  N = 
\begin{vmatrix}
  \vec{i}&\quad \vec{j}&\quad \vec{k} \\
  \pd{x}{u}&\quad \pd{y}{u}&\quad \pd{z}{u} \\
  \pd{x}{v}&\quad \pd{y}{v}&\quad \pd{z}{v} \\
\end{vmatrix}
  \end{aligned}
  \]

  So;\quad $T_{(x_0,y_0,z_0)}S \perp N(x_0,y_0,z_0),\quad ||N(x_0,y_0,z_0)|| = || \pd{r}{u}(u_0,v_0) \times \pd{r}{v}(u_0,v_0)||$
\end{definition}

\begin{example}
..

\begin{center}
\includegraphics[width=0.8\textwidth]{piece_wise_smooth_surface_ex01.jpg}
\end{center}
\end{example}

\begin{definition}
  \underline{the area of a smooth surface}

  Let $\quad S: r: (u,v) \mapsto r(u,v) $ over $D \subset R$ be a smooth parametrized surface the area of S b 
  \[
    \begin{aligned}
      A(S) &= \iint_D ||r_u(u,v)\times r_v(u,v)|| = \iint_{\sigma} d\sigma  \\
       &= ||N(u,v)||dudv
    \end{aligned}
  \]
\end{definition}

\begin{example}
...  
\begin{center}
\includegraphics[width=0.8\textwidth]{area_wise_smooth_surface_ex01.jpg}
\end{center}
\end{example}

\underline{\textbf{Remark:}}
\[S = Gr_f = \{(x,y,z) \in \R^3 : (x,y) \subset D \text{ and } z = f(x,y)\}\]
\[ S : r (u,v) = u\vec{i}+v\vec{j}+f(u,v)\vec{k}\]
\[ d\sigma = \sqrt{1 +\left(\pd{f}{u}(u,v)\right)^2 + \left(\pd{f}{v}(u,v)\right)^2}\]

\[A(S) = \iint_D \sqrt{1 +\left(\pd{f}{u}(u,v)\right)^2 + \left(\pd{f}{v}(u,v)\right)^2} dudv\]

\subsection{Surface integral}

\begin{definition}
  Let  $S: r(u,v) \mapsto x(u,v)\vec{i} + y(u,v)\vec{j} + z(u,v) \vec{k}$ be a parametrized
  smooth surface and $f: \R^3 \to \R$ a function continuous over S, denoted by 
  $\int_S fd\sigma$ is the real number 


  $\iint fd\sigma = \iint_D f(r(u,v))||r_u(u,v)\times r_v(u,v) ||dudv$

  $\iint fd\sigma = \iint_D f(x(u,v)+ y(u,v)\ + z(u,v))||\pd{r}{u}(u,v)\times \pd{r}{v}(u,v) ||dudv$

  \underline{\textbf{Remark:}}

  1) If 
\[
  \begin{aligned}
    S &= \{(x,y,z) \in \R^3 : z = \varphi(x,y)\} \to  \iint_S fd\sigma \\
      &=\iint_D f(x,y,\varphi(x,y))\sqrt{1 +\left(\pd{\varphi}{x}(x,y)\right)^2 + \left(\pd{\varphi}{y}(x,y)\right)^2} dxdy
  \end{aligned}
\] 
  2) If 
\[
  \begin{aligned}
    S &= \{(x,y,z) \in \R^3 : y = \varphi(x,z)\} \to  \iint_S fd\sigma \\
      &=\iint_D f(x,\varphi(x,y),z)\sqrt{1 +\left(\pd{\varphi}{x}(x,z)\right)^2 + \left(\pd{\varphi}{z}(x,z)\right)^2} dxdz
  \end{aligned}
\] 
  3) If 
\[
  \begin{aligned}
    S &= \{(x,y,z) \in \R^3 : x = \varphi(y,z)\} \to  \iint_S fd\sigma \\
      &=\iint_D f(\varphi(y,z),y,z)\sqrt{1 +\left(\pd{\varphi}{y}(y,z)\right)^2 + \left(\pd{\varphi}{z}(y,z)\right)^2} dydz
  \end{aligned}
\] 
\end{definition}

4)
\[S = \{(x,y,z) \in \R^3 : z =c , (x,y) \in D\}; A(S) = A(D) = \iint_D dxdy\]

\begin{example}
  1)...
\begin{center}
\includegraphics[width=0.8\textwidth]{Surface_int_ex01.jpg}
\end{center}

2)..
\begin{center}
\includegraphics[width=0.8\textwidth]{Surface_int_ex02.jpg}
\end{center}
\end{example}


\subsection{Flux of vector field over surface}
\begin{definition} 
  \underline{\textbf{Unit norma vector fields - Oriened surfaces }}

  Let S be a smooth parametrized surface 

  S is regular if  $||r_u\times r_\theta||(u,v) \neq 0,\forall(u,v) \in D \mid D \text{ connected}$

  So, S can be oriented by $N(x,y,z) = \dfrac{1}{||r_u\times r_v||(u,v)}((r_u \times r_v)(u,v)) \begin{cases}
  x = x(u,v) \\y = y(u,v)\\z = z(u,v)\end{cases}$

  or, S can be also oriented by $(-N)(x,y,z) = \dfrac{1}{||r_v\times r_u||(v,u)}((r_u \times r_v)(u,v)) \begin{cases}
  x = x(u,v) \\y = y(u,v)\\z = z(u,v)\end{cases}$
\end{definition}
\begin{example}
  1)
\begin{center}
\includegraphics[width=0.8\textwidth]{Unit_normal_vect_ex01.jpg}
\end{center}

2)
\begin{center}
\includegraphics[width=0.8\textwidth]{Unit_normal_vect_ex02.jpg}
\end{center}
\end{example}

\begin{definition}
  \underline{\textbf{Flux of vector field over regular parametrized }}

  Let $F = P\vec{i}+Q\vec{j}+R\vec{k}$ be a continuous vector field over an open set $\Delta \subset S$
  
  If S is smooth

  \[Flux_S(F) = \iint_D(F\cdot N)(x(u,v),y(u,v),z(u,v)) ||\pd{r}{u}(u,v)\times\pd{r}{v}||du dv\]
  \[= \iint_S(F.N) d\sigma\]
\end{definition}

\textbf{Remark}
$F = P\vec{i}+Q\vec{j}+R\vec{k}$ $S =Gr_f =\{ (x,y,z) \in \R^3 \quad :(x,y)\in D\quad; z = g(x,y)\}$

\[
  Flux_S(f) = 
\begin{cases}
  \iint_D\left[R - \pd{g}{x}(x,y)P - \pd{g}{y}(x,y) Q\right](x,y,g(x,y))dx dy 
  \\ \quad \quad \text{if S is oriented upward} \\
  \iint_D \left[\pd{g}{x}(x,y)P((x,y,g(x,y)) + \pd{g}{y}(x,y)Q(x,y,g(x,y))- R(x,y,g(x,y)\right] dx dy 
  \\\quad\quad \text{ if S is oriented downward}
\end{cases}
\]
\begin{example}
  1)
\begin{center}
\includegraphics[width=0.8\textwidth]{vect_flux_ex01.jpg}
\end{center}
\end{example}
\textbf{Remark}

\[\iint_S(F.N)d\sigma = \iint_D F(x(u,v),y(u,v),z(u,v)).(\pd{r}{u}\times \pd{r}{u} \times \pd{r}{v}(u,v))dudv\]
\[\iint_D 
  \begin{aligned}
  \begin{vmatrix}
    P(x(u,v),y(u,v),z(y,v))&\quad Q(x(u,v),y(u,v),z(u,v)) &\quad R(x(u,v),y(u,v),z(u,v)) \\
    \pd{x}{u}(u,v)&\quad\pd{y}{u}(u,v) &\quad \pd{z}{u}(u,v))\\
    \pd{x}{v}(u,v)&\quad \pd{y}{v}(u,v)&\quad \pd{z}{v}(u,v) \\
  \end{vmatrix}
  \end{aligned}du dv
\]

\begin{example}
  1 ..
\begin{center}
\includegraphics[width=0.8\textwidth]{vect_flux_ex02.jpg}
\end{center}
\end{example}
There are two methods to compute the flux of vector field over a surface S .

1) if S is boundary of a solid domain $\Omega \subset \R^3, (i-e S=\partial D )$, 

(S is closed. without boundary $\partial S = \phi$)
we can  use the divergence theorem

2) If S is not closed, $C = \partial S$ is piece-wise curve we can use Stokes's theorem

\begin{theorem}
  \underline{\textbf{Divergence theorem (Ostrograsky's theorem)}} 

Let $\Omega$ be a closed solid bounded  region in $\R^3$ by 

closed smooth piece-wise regular surface $S = \partial \Omega$ oriented outward from $\Omega$

If $F = P\vec{i}+Q\vec{j}+R\vec{k}$ is of class $C^1$ over $\Omega$ then;

\[Flux_S(F) = \iiint div(F) dv = \iiint_\Omega\left(\pd{P}{x} + \pd{Q}{y} + \pd{R}{Z}\right)(x,y,z) dx dy dz\]
\end{theorem}

\begin{theorem}
  \underline{\textbf{Stockes's theorem}}
  Let S be orented by N ; $C = \partial S$ is a piece-wise smooth simple closed curve with positive orientation. If $F = P\vec{i}+Q\vec{j}+R\vec{k}$ is $C^1$ vector field over a open region $\Delta \subset S,$ then:

  \[\oint_{C = \partial S}F\cdot dr = \iint_S(Curl(F)\cdot N)d\sigma = Flux_S(Curl(F))\]
\end{theorem}

\begin{example}
  1)
\begin{center}
\includegraphics[width=0.8\textwidth]{Stockes_ex01.jpg}
\end{center}
\end{example}

\begin{example}
  stokes example
\begin{center}
\includegraphics[width=0.8\textwidth]{Stokes_ex1.jpg}
\end{center}

\begin{center}
\includegraphics[width=0.8\textwidth]{Stokes_ex2.jpg}
\end{center}
\end{example}
\end{document}


