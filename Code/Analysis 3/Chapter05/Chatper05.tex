\documentclass[12pt]{article}
\usepackage[utf8]{inputenc}
\usepackage[T1]{fontenc}
\usepackage{amsmath, amssymb, amsthm}
\usepackage{siunitx}
\usepackage{tikz}
\usepackage{pgfplots}
\usepackage{geometry}
\usepackage{titlesec}
\usepackage{graphicx}

\geometry{a4paper, margin=1in}
\pgfplotsset{compat=1.18}
\usetikzlibrary{3d, arrows.meta}

% Theorem environments
\newtheorem{definition}{Definition}
\newtheorem{proposition}{Proposition}
\newtheorem{remark}{Remark}
\newtheorem{example}{Example}
\newtheorem{theorem}{Theorem}

% Paragraph formatting
\setlength{\parindent}{1.5em}
\setlength{\parskip}{1em}

% Add extra vertical space
\titleformat{\section}{\normalfont\Large\bfseries}{\thesection}{1em}{}
\titlespacing*{\section}{0pt}{2ex plus 1ex minus .2ex}{1ex plus .2ex}

\title{Chapter 04: Sequences of Functions}
\author{Notes from Prof. Zeglaoui's Course}

\hbadness=10000
\hfuzz=\maxdimen

\newcommand{\N}{\mathbb{N}}
\newcommand{\R}{\mathbb{R}}
\newcommand{\Z}{\mathbb{Z}}
\newcommand{\Q}{\mathbb{Q}}
\newcommand{\C}{\mathbb{C}}
\newcommand{\E}{\mathbb{E}}
\newcommand{\F}{\mathbb{F}}

\newcommand{\pd}[2]{\dfrac{\partial #1}{\partial #2}}
\newcommand{\pdd}[3]{\dfrac{\partial^2 #1}{\partial #2\,\partial #3}}
\newcommand{\pddx}[2]{\dfrac{\partial^2 #1}{\partial #2^2}}
\newcommand{\dd}[2]{\dfrac{\mathrm{d} #1}{\mathrm{d} #2}}
\newcommand{\ddd}[2]{\dfrac{\mathrm{d}^2 #1}{\mathrm{d} #2^2}}
\newcommand{\diff}{\,\mathrm{d}}

\newcommand{\vect}[1]{\begin{pmatrix}#1\end{pmatrix}}

\begin{document}

\maketitle

\vspace{2em}

\section{Sequences of Functions}

\subsection{Convergence of a Sequence of Functions}

\begin{definition}
Let \(D\) be a subset (non-empty) of \(\mathbb{R}\). We denote by \(\mathcal{F}(D,\mathbb{R})\) the \(\mathbb{R}\)-vector space of real functions from \(D\to \mathbb{R}\).

A \textbf{sequence of functions} over \(D\) is a sequence in \(\mathcal{F}(D,\mathbb{R})\), i.e.:
\[
\mathbb{N}\to \mathcal{F}(D,\mathbb{R}), \quad n\mapsto f_n:\begin{cases}
D\to \mathbb{R}\\
x\mapsto f_n(x)
\end{cases}
\]
It is often denoted by \((f_n)\) or \((f_n)_{n\in \mathbb{N}}\).
\end{definition}

\begin{definition}
Let \((f_n)\) be a sequence of functions over \(D\). We say that \((f_n)\) \textbf{pointwise converges}, or simply \textbf{converges}, to \(f\in \mathcal{F}(D,\mathbb{R})\) if:
\[
\forall x\in D,\; \lim_{n\to +\infty}f_n(x) = f(x)
\]
or equivalently, \((f_n) \xrightarrow{\mathrm{pointwise}} f\).

In logical form:
\[
\forall x\in D,\;\forall \epsilon >0,\;\exists N = N(\epsilon ,x):\forall n\geq N,\;|f_n(x) - f(x)|< \epsilon.
\]
\end{definition}

\begin{example}
...

\begin{center}
--\includegraphics[width=0.8\textwidth]{geometric_sequence.jpg}
\end{center}

\end{example}

% IMAGE PLACEHOLDER 1 (from Page 2)
% \begin{figure}[h]
% \centering
% \includegraphics[width=0.8\textwidth]{image1.png}
% \caption{Illustration of convergence}
% \end{figure}

\begin{definition}[Uniform Convergence]
We say that \((f_n) \xrightarrow{C_V} f\) if for all \(n \in \mathbb{N}\), \((f_n - f)\) is bounded over \(D\) and
\[
\lim_{n \to +\infty} \left( \sup_{D} |f_n - f| \right) = 0.
\]
Equivalently:
\[
\forall \epsilon >0,\;\exists N = N(\epsilon):\forall n\geq N,\;\forall x\in D,\;|f_n(x) - f(x)| < \epsilon.
\]
\end{definition}

\begin{remark}
Uniform convergence implies pointwise convergence: \((f_n) \xrightarrow{C_V} f \Rightarrow (f_n) \xrightarrow{P_V} f\).
\end{remark}

\begin{example}
...
\begin{center}
--\includegraphics[width=0.8\textwidth]{Uniform_ex_part1.jpg}
\end{center}

\begin{center}
--\includegraphics[width=0.8\textwidth]{Uniform_ex_part2.jpg}
\end{center}
\end{example}

\subsection{Properties of Uniformly Convergent Sequences of Functions}

\begin{theorem}[Continuity]
If
\[
\begin{cases}
\exists N\in \mathbb{N} \text{ such that } \forall n \geq N,\; f_n \text{ is continuous on } D,\\
\text{and } (f_n) \xrightarrow{C.V} f \text{ on } D,
\end{cases}
\]
then \(f\) is continuous on \(D\).

In other words:
\[
\forall x_0\in D,\; \lim_{x\to x_0} f(x) = \lim_{x\to x_0} \left( \lim_{n\to +\infty} f_n(x) \right) = \lim_{n\to +\infty} \left( \lim_{x\to x_0} f_n(x_0) \right) = f(x_0).
\]
\end{theorem}

\begin{remark}
If \((f_n) \xrightarrow{P.C} f\) and there exists a sequence \((x_n) \subset D\) such that
\[
\lim_{n\to +\infty} [f_n(x_n) - f(x_n)] \neq 0,
\]
then \((f_n)\) does not converge uniformly to \(f\) on \(D\), because:
\[
(f_n) \xrightarrow{C.V} f \implies \forall (x_n) \subset D,\; \lim_{n\to +\infty} [f_n(x_n) - f(x_n)] = 0.
\]
\end{remark}

\begin{example}
...

\begin{center}
--\includegraphics[width=0.8\textwidth]{4_1_2_ex01.jpg}
\end{center}
\end{example}

% IMAGE PLACEHOLDER 2 (from Page 3)
% \begin{figure}[h]
% \centering
% \includegraphics[width=0.8\textwidth]{image2.png}
% \caption{Graphical illustration}
% \end{figure}

\begin{theorem}[Integrability]
If \(f_n:[a,b]\to \mathbb{R}\) is Riemann-integrable for all \(n\in \N\), and \((f_n) \xrightarrow{C.V} f\) on \([a,b]\), then:
\[
\lim_{n\to +\infty} \int_a^b f_n(x) \diff x = \int_a^b f(x) \diff x = \int_a^b \left( \lim_{n\to +\infty} f_n(x) \right) \diff x.
\]
\end{theorem}

Proof:
\[
\begin{array}{r l r}
\left| \int_{a}^{b} f_n(x) \diff x - \int_{a}^{b} f(x) \diff x \right| 
&= \left| \int_{a}^{b} (f_n(x) - f(x)) \diff x \right| \\
& \leq \int_{a}^{b} |f_n(x) - f(x)| \diff x \\
& \leq \left( \sup_{[a,b]} |f_n - f| \right) (b-a) \xrightarrow[n\to+\infty]{} 0.
\end{array}
\]

\begin{example}
...

\begin{center}
--\includegraphics[width=0.8\textwidth]{integrability_part1.jpg}
\end{center}

\begin{center}
--\includegraphics[width=0.8\textwidth]{integrability_part2.jpg}
\end{center}
\end{example}

% IMAGE PLACEHOLDERS 3 & 4 (from Page 4)
% \begin{figure}[h]
% \centering
% \includegraphics[width=0.8\textwidth]{image3.png}
% \caption{First example figure}
% \end{figure}

% \begin{figure}[h]
% \centering
% \includegraphics[width=0.8\textwidth]{image4.png}
% \caption{Second example figure}
% \end{figure}

\begin{theorem}[Differentiability]
Let \(I\) be an interval and \((f_n)_n\) a sequence of class \(C^1\) functions on \(I\) such that:
\[
(f_n')_n \xrightarrow{C.V} g \quad \text{and} \quad \exists x_0 \in I \text{ such that the numerical sequence } f_n(x_0) \text{ converges to } l \in \R.
\]
Then \((f_n)_n\) converges uniformly to a function \(f\) on \(I\), where \(f\) is defined by:
\[
\begin{cases}
f' = g,\\
f(x_0) = l.
\end{cases}
\]
Thus:
\[
\lim_{n\to +\infty} f_n'(x) = g(x) = f'(x) = \left( \lim_{n\to +\infty} f_n(x) \right)'.
\]
\end{theorem}

\textbf{Proof:}
\[
f(x) = l + \int_{x_0}^x g(t) \diff t, \quad \forall x\in I.
\]
Recall that:
\[
f_n(x) = f_n(x_0) + \int_{x_0}^{x} f_n'(t) \diff t.
\]
Using the integrability theorem, we get:
\[
(f_n) \xrightarrow[I]{P.C} f,
\]
where P.C means pointwise continuous.

\section{Convergence of series of functions}

\begin{definition}[Series of functoin - convergence domain]
  Let $(f_n)_n$ be a sequence of functions on $D$, When we consider the family of numerical series 
  $\sum_{n \in \N}f_n(x)$, parametrized by $D$, we speak about  the series of functions $\sum_{n \in \N}f_n$

  By convergence domain of $\sum f_n$ (point-wise convergence) , we mean the set 

  $\{x \in D;\quad \sum_{n \in \N} f_n(x) \text{ converges}\}$

  the sum of $\sum_{n \in \N} f_n$ is defined on the domain of convergence  by:

  \[x \mapsto \sum_{n = 0}^{+\infty}=F(x) = \lim\limits_{n \to +\infty} F_n(x) \]

  where 
  \[
  F_n = f_0 + f_1 + \dots + f_n  = \sum_{k = 0}^n f_k \]
  \[
    R_n = F- F_n = \sum_{k = n +1}^{+\infty} f_k \quad, \forall n \in \N
  \]

\end{definition}
  \begin{example}

    The geometric series $\sum_{n \in \N} x^n$

    the geometric series is pointe-wise convergente  on $\left[-1,1\right]$
    and its sum :
    \[ x \mapsto \sum_{n = 0}^{+\infty} x^n = \frac{1}{1-x}\quad, \forall a \in \left[-1,1\right]\]

\begin{center}
--\includegraphics[width=0.8\textwidth]{covnergence_domain_ex01.jpg}
\end{center}
  \end{example}
  
  \begin{definition}[Uniform convergence and absolute convergence]

   1) We say that $\sum_{n \in \N} f_n$ is absolutly convergent on $D$ if $\sum_{n \in \N}|f_n|$  
    is pointwise convergent

    2) We say that $\sum f_n$ is uniformly convergent if, $(R_n) \xrightarrow[D]{C.U} 0 $ (C.U $\to$
    uniformly converges
  \end{definition}

  \textbf{Remark:} 
  $\sum f_n$ C.U on $D \implies (f_n)\xrightarrow[D]{C.U} 0$ for example $\sum_{n \in \N} x^n$ 
  does not uniformly CV on $\left[-1,1\right]$  

  \textbf{Remark:} If there exists a real sequence $(x_n) \subset D$ such that the numerical serie
  $\sum_{n \in \N}f_n(x_n)$ diverge then $\sum f_n$ does not uniformly CV on $D$ 
for example $\sum_{n \in \N}ne^{-nx}$ is point-wise Cv on $\left] 0, +\infty\right]$
but not uniformly convergent because $x_n = \dfrac{1}{n}$ 
\[f_n(x_n) = ne^{-n\frac{1}{n}} = \frac{n}{e} \xrightarrow[n \to +\infty]{}  0
\implies \sum f_n(\frac{1}{n}) \text{ div}  \]
\[ \implies \sum f_n \text{ does not uniformly Cv on $ \left[ 0, +\infty\right]$}\]

\begin{theorem}[Abel's criterion for uniform convergence]
  
  Let $f_n = a_n b_n\quad,\forall n \in \N$; $(a_n)(b_n) $ sequences of functions such that :

  1) $\forall x \in D,\quad (b_n(x))_{n\in \N}$ is positive, decreasing  and $(b_n) \xrightarrow[D]{C.U}$ 0

  
  2)
  \[
    \exists M \geq 0, \forall x \in D; |\sum_{k=0}^n a_k(x)|\leq M;\quad \forall n \in \N
  \]
  then $\sum f_n$ Converges uniformly  on D
\end{theorem}

\begin{example}
..
\begin{center}
\includegraphics[width=0.8\textwidth]{Abel_ex01.jpg}
\end{center}

\end{example}

\begin{definition}[Normal Convergence]

  the series $\sum F_n$ is said to be normally convergent over $D$ if the positive numerical serie
  $\sum_{n \in \N} sup_{x \in D}|f_n(x)| $ Converge 
\end{definition}

\textbf{Remark:}

\begin{center}
\includegraphics[width=0.8\textwidth]{convergence_types_Remark.jpg}
\end{center}

\begin{theorem}
  Let $\sum f_n$ be a series of functions on $D$ if there exists a real positive sequence 
  $(V_n)_{n \in \N}$, such that: 

  1) $\sum  f_n  \text{N.C on D (normally convergent) }$ 
  2) $\forall x \in D, \forall n \in \N;\quad |f_n(x)| \leq V_n$

  then $\sum f_n$ N.C on D

  \begin{example}
    \[
      \sum \frac{cos(nx)}{n^\alpha} \text{ and } \sum \frac{sin(nx)}{n^\alpha} \text{ are  N.C} \Leftrightarrow 
      \alpha > 1
    \]

    Because $\forall x \in \R; \forall n \in \N$;\quad $|\dfrac{cos(nx)}{n^\alpha}| \leq \dfrac{1}{n^\alpha}$ and
    $| \dfrac{sin(nx)}{n^\alpha}| \leq \dfrac{1}{n^\alpha}$
  \end{example}

  \section{Properties of uniformly convergent series theorem(continuty):}

  Let $\sum f_n$ be a series of contiuous functions that convergent uniformly on $D$, its sum is also 
  continuous on $D$  . then :
 \[
   \forall a \in D; \sum_{n = 0}^{+ \infty} f_n(a) = \sum_{n = 0}^{+\infty}\left(\lim\limits_{x \to a} f_n(x)\right) 
   = \lim\limits_{x \to a}\left(\sum_{n = 0}^{+ \infty} f_n(x) \right)
 \]
\end{theorem}

\begin{example}
  the function $f: x \mapsto \sum_{n =1}^{+\infty} \dfrac{1}{n^2 + x^2}$

  is continuous on $\R$
\end{example}

\begin{theorem}[Integrability]   

  Let $\sum_{n \in \N}f_n$ be uniformly convegent on $\left[a , b \right]$,where $f_n$ is
  Riemann-integrable on $\left[a , b \right]$, then the sum is Riemann-integrable on $\left[a , b \right]$ and 
  \[
    \int_a^b \left( \sum_{n = 0}^{+\infty} f_n(x)\right)dx = \sum_{n = 0}^{+\infty}\left( \int_a^b f_n(x) dx\right)
  \]
\end{theorem}

\begin{example}
..
\begin{center}
\includegraphics[width=0.8\textwidth]{Uni_Cv_properties_ex.jpg}
\end{center}
\end{example}

\begin{theorem}[derivability]

  Let $\sum f_n$ defined  on an interval $I \subset R$
  such that :
  1) $f_n$ is of class $C^1$ on I; $\forall n \in \N$

  2) $\sum_{n\in \N} f'_n $ U.C on I with sum G 

  3) $\exists x_0 \in I |$ the numerical series $\sum_{n \in \N} f_n(x_0)$ convergence, then 
  $\sum f_n$ is pointwise convergent on I, with sum F where  F is defined by F' =  G 
  and $F(x_0) = \sum_{n = 0 }^{+\infty} f_n(x_0)$
\end{theorem}

 \textbf{Remark:}

\begin{center}
\includegraphics[width=0.8\textwidth]{Remark_Derivability.jpg}
\end{center}

\end{document}
