\documentclass[12pt]{article}

\usepackage[utf8]{inputenc}
\usepackage[T1]{fontenc}
\usepackage{amsmath, amssymb, amsthm}
\usepackage{geometry}
\geometry{a4paper, margin=1in}

% Optional for colored boxes
\usepackage{tcolorbox}

% Theorem environments
\newtheorem*{defn}{Definition}

% Paragraph formatting
\setlength{\parindent}{1.5em} % first line indent
\setlength{\parskip}{0.5em}   % space between paragraphs

\title{Chapter 01: Functions of Several Variables}
\author{Author: Who's Who (Them)}
\date{October 2025}

\begin{document}

\maketitle

\section*{Functions of Several Variables}

Let $E, F$ be two sets.  
The product
\[
E \times F = \{ (x,y) \mid x \in E \text{ and } y \in F \}.
\]

\begin{defn}
A relation $R$ from $E$ to $F$ is a given subset $G$ (or $G_R$) of $E \times F$.  

We say that $y$ is the \textbf{image} of $x$ ($x$ is the \textbf{pre-image} of $y$) if $(x,y) \in G$.
\end{defn}

\section*{Inverse Relation}

The inverse relation $R^{-1}$ of $R$ is from $F$ to $E$ defined by
\[
y R^{-1} x \Leftrightarrow x R y
\]

\[
G_{R^{-1}} = s(G_R), \quad \text{where } s =
\begin{cases}
E \times F \to F \times E,\\
(x,y) \mapsto (y,x)
\end{cases}
\]

\begin{defn}
A function $f : E \to F$ is a relation from $E$ to $F$ such that each $x \in E$ has at most one image $y \in F$.  

The \textbf{domain} of $f$ is
\[
Dom(f) = \{ x \in E \mid \text{the image of $x$ exists} \}.
\]

If $x \in Dom(f)$, we denote $f(x)$ its image.  
When $Dom(f) = E$, $f$ is called a \textbf{map} from $E$ to $F$.

We define
\[
G_F = \{ (x,y) \in E \times F \mid x \in Dom(f),\ y = f(x) \}
\]

\[
\text{If } F = \mathbb{R}, \quad f : E \to \mathbb{R} \text{ is a real function}
\]

\[
\text{If } F = \mathbb{C}, \quad f : E \to \mathbb{C} \text{ is a complex function}
\]
\end{defn}

\end{document}
