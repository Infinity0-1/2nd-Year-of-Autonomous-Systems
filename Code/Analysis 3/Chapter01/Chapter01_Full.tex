\documentclass[12pt]{article}
\usepackage[utf8]{inputenc}   % handle accented characters
\usepackage[T1]{fontenc}      % font encoding
\usepackage{amsmath, amssymb, amsthm} % math symbols
\usepackage{siunitx}           % optional
\usepackage{tikz}              % optional for diagrams
\usepackage{geometry}
\geometry{a4paper, margin=1in}


\newtheorem*{defn}{Definition}

% Paragraph formatting
\setlength{\parindent}{1.5em} % first line indent
\setlength{\parskip}{1em}   % space between paragraphs

\title{Chapter01: functions of Several variables}
\author{author: who's who(them)}
\date{a day}



\hbadness=10000
\hfuzz=\maxdimen

\newcommand{\N}{\mathbb{N}}
\newcommand{\R}{\mathbb{R}}
\newcommand{\Z}{\mathbb{Z}}
\newcommand{\Q}{\mathbb{Q}}
\newcommand{\C}{\mathbb{C}}
\newcommand{\E}{\mathbb{E}}
\newcommand{\F}{\mathbb{F}}

% --- Derivatives (display style) ---
\newcommand{\pd}[2]{\dfrac{\partial #1}{\partial #2}}
\newcommand{\pdd}[3]{\dfrac{\partial^2 #1}{\partial #2\,\partial #3}}
\newcommand{\pddx}[2]{\dfrac{\partial^2 #1}{\partial #2^2}}
\newcommand{\dd}[2]{\dfrac{\mathrm{d} #1}{\mathrm{d} #2}}
\newcommand{\ddd}[2]{\dfrac{\mathrm{d}^2 #1}{\mathrm{d} #2^2}}
\newcommand{\diff}{\,\mathrm{d}}

%% vertical vect macro
\newcommand{\vect}[1]{\begin{pmatrix}#1\end{pmatrix}}
\begin{document}

\maketitle
\section{Functions of several variables}


Let $E, F$ be two sets.\\
$\pd{f}{x}$
The product $E \times F = \{(x, y) \mid x \in E \text{ and } y \in F\}$.
\subsection*{\underline{Def:}}
Relation $R$ from  $E$ to $F$ is a given subset ($G$ or $G_{R}$) from $E \times F$ \\
We say that $y$ is the image of $x$ ( $x$ is the pre-image of $y$) \\
if  $(x,y) \in G$

\section*{Inverse Relation}
the relation inverese $R^{-1}$ of $R$ is from $F$ to $E$ defined by: 
\[
yR^{-1}x  \Leftrightarrow  xRy 
\]
\[
  G_{R^{-1}} = s(G_R) , \text{where } s =
  \begin{cases}
    E \times F \to F\times E \\
    (x,y) \to (y,x)
    \end {cases}
 \] 
 \subsection*{\underline{Def:}}
 A function $f : E \to F $ is a relation from  $E$ to $F$ such that :\\
 each $x \in E$, there exist atmost an image $ y \in F$\\
 the domain of $f : Dom(f)$ is the subset:
 \[
   Dom(f) = \{x \in E\mid \text{the image of $x$ exists\}}
  \]
  if $ x \in Dom(f) $, we denote $f(x)$ its image. \\
  when $Dom(f) = E$, $f$ is a map from $E$ to $F$
  \[
    \begin{aligned}
    G_F = \{ (x,y) \in E\times F\mid x \in Dom(f) \text{ and } y = f(x)\} \\
    \text{If } F=R : f: E \to R \text{ is a real function} \\
    \text{If } F=C : f:E \to C \text{ is a complex function}
  \end{aligned}
  \]

\begin{defn}
  Let $n,m \in \mathbb{N}$, a function \quad $\mathbb{R}^n \to \mathbb{R}^m$ is called a function of n variables :

\[
  f: \quad \mathbb{R}^n \to \mathbb{R}^m
\]
\[
  (f_1(x_1,\dots,x_n),\dots,f_n(x_1,\dots,x_n))
\]
%We note $f_j$ $Pr_j$ of(Pr abbreviation to projection)  :  the next not is from chatgpt
We denote by $f_j$ the $j$-th component (or projection $Pr_j$) of $f$:
\[
\mathbb{R}^n \to \mathbb{R}
\]
\[
 (x_1,\dots,x_n) \to f_j(x_1,\dots,x_n)
\]
another way to write it :
\[
  Pr_j : \quad \forall j \in \{1,\dots,n\}
\]
\[
  (y_1,y_2,\dots,y_j,\dots,y_m) \mapsto y_j
\]
\[
  f = (f_1,\dots,f_m) = \sum_{j = 1}^m f_j e_j \quad/\quad e_j = (0,0, \dots,1,\dots,0)
  \]

  If $m = 1$, $f$ is a real function \par 
  If $m > 1$, $f$ is a vector function \par
  the range(of the image) of $f$ is the subset of $\R^n$ defined by:
  \[
    range(f) = \{y \in \R^m \mid \exists x \in Dom(f) , y =f(x) \}
  \]
\end{defn}

\subsection*{\underline{Example:}}
1) $L(\R^n, \R^m)$ the set of linear maps from $\R^n \text{ to } \R^m$ 
$L(\R^n, \R^m \subset F(R^n,R^m)$ \\
where $F(R^n,R^m)$ is the set of all functions from  $\R^n \text{ to } \R^m$ \par

2) An affine function:
\[f: \R^n \to \R^m\]
\[ x \to A_{(m,n)}x + b\]
\[\text{where:   }b \in \R^m \quad \text{ and } A \in M_{(m,n)}(R)\]


\[f: \R^2 \to \R\]
\[ (x,y) \to (x,y)\vect{\alpha \\ \beta}\]


\section{\underline{\textbf{Operations on functions of several variables}}}
1) $\forall f,y\in F(\R^n,R^m), \forall_{\alpha,\beta} \in \R$
\[\alpha f+\beta y : \R^n \to \R^m\]
\[x \mapsto \alpha f(x) + \beta y(x)\]
$(F(\R^n,\R^m);+;.)$ is a vector space

2) If $f \in F(\R^n,\R^m)$ and $g \in F(\R^n,\R^m)$ then:

$g \circ f \in F(\R^n,\R^m)$ defined by:
\[\R^n \xrightarrow{f} \R^m \xrightarrow{g} \R^p\]
\[x \mapsto f(x) \mapsto g(f(x))\]

3) Let $f,y \in F(\R^n,\R^m)$, two scalar functions.

The product $f \cdot g \in F(\R^n,\R^m)$ defined by:
\[(f \cdot g) (x) = f(x) \cdot g(x), \forall_x\in Dom(f) \cap Dom(g)\]
if $g(x) \neq 0$, $\forall_x \in Dom(g), \dfrac{1}{g}: x \to \dfrac{1}{g}$
\[Dom(\frac{1}{g}) \ \{x \in Dom(g) \mid g(x) \neq 0\}\]
and $\dfrac{f}{g} \Leftrightarrow f\cdot (\dfrac{1}{g})$

\subsection*{\underline{Example:}}
1) A monomial function of degree P is :
\[(x_1,\dots,x_n) \to \alpha x_1^{p_1}\cdot x_2^{p_2}\dots x_n^{p_n}\]
\[P_1+ P_2 + \dots P_m = P\]

2) \underline{An homogeneous polynome:}

is a function of degree P is a finite Sum of monomial functions of degree P.


3) \underline{Polonomial function :}
polyonmial function is a  finite sum of homogeneous polonomial functions

4) \underline{Rationl Function:}
A rational function is the quotient of two polynomial functions \\ \\ 


\begin{defn}
  the graph function of f from $\R^n \to \R^m$ :

  Let $f:\R^n \to \R^m$ and $Dom(f)$ is the domain of the graph $G_f$ of $f$ is the subset of $\R^n + R^m \cong R^{n + m}$ defined by :
  \[ G_fr = \{(x,y) \in \R^n \times \R^m : x \in Dom(f) \text{ and } y = f(x)\}\]
\end{defn}

\subsection*{\underline{Remark}}
if we consider the function
\[f: Dom(f) \to \R^{n+m}\]
\[ x \mapsto (x,f(x))\]

then the $range(\tilde{f}) = Gr_{\tilde{f}} $

$\tilde{f}$ is parametrization of $Gr_f$

\subsection*{\underline{Examples:}}
1)

Affine function
\[f\colon \R \to \R\]
\[ x \mapsto ax + b\]
(graph)
\[Gr_f = \{(x,y) \in \R^2 : x \in R \text{ and } y = ax + b\}\]

\[f\colon \R \to \R^2\]
\[ x \mapsto (x,ax + b)\]

2)

Let $A(x_a, y_a, z_a)$; $C(x_c,y_c,z_c)$ two points in the space $(\R^3)$
\[f\colon \R \to \R^2\]
\[ x \mapsto (x,ax + b)\]

We will get a function:

\[f\colon \R^2 \to \R\]
\[ (x,y) \mapsto f(x,y)\]
\begin{align*}
  \vect{x - x_a \\ y- y_a \\ z-z_a} &=t\vect{x - x_c \\ y- y_c \\ z-z_c} \\
  \Leftrightarrow &\begin{cases}
    x = x_a + t(x_c -x_a)\\
    y = y_a + t(y_c -y_a)\\
    z = z_a + t(z_c -z_a)
  \end{cases}\\
\end{align*}

\[
\tilde{f} = \R \to \R^3
\]
\[
  t \mapsto (x(t),y(t),z(t))
\]
If $x_c \neq x_a$ then $t = \dfrac{x - x_a}{x_c -x_a}$
\[
\begin{cases}
y = y_a +  \dfrac{x - x_a}{x_c -x_a}(y_c -y_a) \\
z = z_a +  \dfrac{x - x_a}{x_c -x_a}(z_c -z_a) \\
\end{cases}
\]
\[
\begin{aligned}
f: \R &\to \R^2 \\
x &\mapsto \bigg(y_a +  \dfrac{x - x_a}{x_c -x_a}(y_c -y_a), z_a +  \dfrac{x - x_a}{x_c -x_a}(z_c -z_a) \bigg)
\end{aligned}
\]
\[f: \R \to \R^2\]
\[x \mapsto (ax+b , cx+d)\]

2)
\[f: \R^2 \to \R\]
\[(x,y) \mapsto \sqrt{16 - 4x^2 -y^2}\]

\[Dom(f) = \{(x,y) \in \R^2 \mid 4x^2 + y^2 \le 16 \}\]
(draw the domain of definition in a plane)

\[range(f) = \{z \in \R \mid \exists(x,y)\in Dom(f) \text{ s.t } z=\sqrt{16 - 4x^2 -y^2} \}\]
\[
\begin{aligned}
  4x^2 + y^2 \ge 0 &\implies -4x^2 - y^2 \le 0 \\
                   &\implies 0\le-4x^2 - y^2 \le 16 \\
                   &\implies 0\le \sqrt{16-4x^2 - y^2} \le 4 \\
\end{aligned}
\]



$z \in range(f) \implies z \in \left[0,4\right]$

$range f \subset \left[0,4\right]$
\[
\begin{aligned}
z \in \left[0,4\right]  \implies \exists (x,y) \in Dom(f) : z =f(x,y) \\
\forall_z \in \left[0,4\right] , \exists (x,y) = (0,\sqrt{16-z^2}) \in Dom(f): z= f(x,y) \\
\end{aligned}
\]

$\text{ then }range(f) = Im(f) = \left[0,4\right]$


\[
\begin{aligned}
  Gr_f &=\{(x,y,z) \in \R^3 \mid (x,y) \in Dom(f) \text{ and } z = sqrt{16 - 4x^2 - y^2} \} \\
  Gr_f &=\left\{(x,y,z) \in \R^3 \mid (x,y) \in Dom(f) \text{ and } \begin{cases}16 = z^2 + 4x^2 + y^2 \\ 
  z \ge 0\end{cases}\right\} \\
\end{aligned}
\]
(draw the surface her in a 3d space)

\section{\underline{Limits and continuity:}}
\begin{defn}
Let $f: \R^n \to \R$ and \quad$l\in \R$ and $a\in \R^n$
\[
\begin{aligned}
  \lim\limits_{x \to a}f(x) &= l \\
                            &\Leftrightarrow \forall\epsilon > 0, \exists \delta > 0, \forall x \in Dom(f); 0< d(x,a)<\delta \implies|f(x) -l| \le \epsilon
\end{aligned}
\]
\[
\begin{aligned}
  \lim\limits_{x \to a}f(x) &= +\infty ( resp -\infty) \\
                            &\Leftrightarrow \forall\alpha > 0, \exists \delta > 0, \forall x \in Dom(f); 0< d(x,a)<\delta \implies f(x) > \alpha (resp f(x) < -\infty)
\end{aligned}
\]
where 
\[ d:\R^n \times \R^n\to\R_{+} = \left[ 0, +\infty\right] \]
\[ (x,y) \mapsto d(x,y)\]
is one of the following:

\textbf{\underline{the euclidian distance}}
\[
\begin{aligned}
  d_2((x_1,\dots,x_n),(y_1,\dots,y_n)) &= \sqrt{(y_1-x_1)^2+\dots+(y_n-x_n)^2} \\ 
  d_1((x_1,\dots,x_n),(y_1,\dots,y_n)) &= |y_1-x_1|+\dots+|y_n-x_n| \\ 
  d_{\infty}((x_1,\dots,x_n),(y_1,\dots,y_n)) &= \{|y_i-x_i|^2+\dots+(y_n-x_n)^2\} \\ 
\end{aligned}
\]
\end{defn}
  If  \quad  $n =1 \implies d_1 = d_2 = d_{\infty}$


\subsection*{\underline{Proposition:}}
\[
  \lim\limits_{x \to a}f(x) = l \Leftrightarrow \forall(x_n) \subset Dom(f) \mid \lim\limits_{n \to +\infty}x_n =a \text{ then } \lim\limits_{n \to \infty}f(x_n) = l
\]

\subsection*{\underline{Remarke}}
  If the limit $l$ exists it is \textbf{unique}

\begin{defn}

  Let 
\[f= (f_1,\dots,f_n): \R^n \to \R^m\]
\[x \mapsto (f_1(x),\dots,f_n(x))\]
\[
\begin{aligned}
  &\forall_{j \in \{1,\dots,m\}} \quad l_j = \lim\limits_{x \to a}f_j(x) \\
  If \quad &\forall_j ; l_j \in \R ; \lim\limits_{x \to a}f(x) = l = (l_1,\dots,l_m)\in \R^m \\
  If \quad &\exists_j; l_j = \pm \infty \quad \lim\limits_{x \to a}f(x) = \infty
\end{aligned}
\]
\end{defn}
therefore: $\lim\limits_{x \to a}f(x)$ doesn't exist


\begin{defn}

$f = (f_1,\dots,f_n): \R^n \to \R^m$
  
$f$ is continous at a point a / $a \in Dom(f)$

$\lim\limits_{x \to a} f(x) = f(a)$ / $x \in Dom(f)$
\end{defn}

\subsection*{\underline{Example:}}

$f$ conitous at $x =a $

$\Leftrightarrow \forall_{(x_n)} \subset Dom(f) \quad / \quad \lim\limits_{n \to +\infty} x_n = a, \lim\limits_{n \to +\infty}f(x_n) = f(a)$

\subsection*{\underline{Remark:}}
To prove that $\lim\limits_{x \to a}f(x)$ doesn't exist.

we can use two methods: \par
\underline{\textbf{$1^{st}$method}}

We can give two sequences $(x_n),(y_n)$ from Dom(f) that converges to a 
$\begin{cases} \lim\limits_{n \to |\infty} x_n =a \\\lim\limits_{n \to |\infty} y_n =a\end{cases}$

and $\lim\limits_{n \to +\infty}f(x_n) \neq \lim\limits_{n \to +\infty}f(x_n)$

\underline{\textbf{$2^{nd}$method}}

We give two paths (Continous maps $\left[0,5\right[$ to Dom(f)) on Dom(f) . \quad  $\gamma(0) = a$
\[
\begin{cases}
\gamma_1 \left[0, \delta_1\right[ \to Dom(f), \gamma_1(0) = a\\
\gamma_2 \left[0, \delta_2\right[ \to Dom(f), \gamma_2(0) = a
\end{cases}
\]
If $ \lim\limits_{t \xrightarrow{>} o}f(\gamma_1(t)) \neq \lim\limits_{t \xrightarrow{>} o}f(\gamma_2(t))$,\quad then limit doesn't exist 


\subsection{\underline{Example:1}}
\[f(x,y) = \frac{x^2 - y^2}{x^2 + y^2}\]
\[
\begin{cases}
\gamma_1: \left[0, +\infty\right[ \to \R^2 \\
  t \mapsto (t,0)\\ \\
\gamma_2: \left[0, +\infty\right[ \to \R^2 \\
  t \mapsto (0,t)\\
\end{cases}
\]

\[f(\gamma_1(t)) = \frac{t^2 -0^2}{t^2 + 0^2}=1\]
\[f(\gamma_2(t)) = \frac{0^2-t^2 }{  0^2+t^2}=-1\]

Since  $ \lim\limits_{t \xrightarrow{>} o}f(\gamma_1(t)) \neq \lim\limits_{t \xrightarrow{>} o}f(\gamma_2(t))$

\subsection*{\underline{\textbf{Properties:}}}
\underline{\textbf{1) The Linearity:}}

\[\lim\limits_{x \to a}f(x) = l_1 \in \R^{n}\text{  and  }\lim\limits_{x \to a}g(x) = l_2 \in \R^{n}\]
\[\text{then} \forall_{\alpha, \beta}\in \R, \lim\limits_{x \to a}\left[\alpha f(x) + \beta f(x)\right] = \alpha l_1 + \beta l_2\]

\underline{\textbf{2) The product}}

\[f\times g : \R^n \to \R\]
\[\text{If }\quad \lim\limits_{x \to a}f(x) = l_1, \text{ and } \lim\limits_{x \to a}g(x) = l_2 \]
\[\text{Then }\quad \text{ Then }\lim\limits_{x \to a}(fg)(x) = l_1l_2\]
\[\text{And If }g(x) \neq 0,\quad \lim\limits_{x \to a}\frac{f(x)}{g(x)} \ = \frac{l_1}{l_2}\]

\underline{\textbf{3)}}

If \quad $f: \R^n \to \R^m$ continous at a and 
\[\lim\limits_{x \to f(x)}g(x) =l \quad / \quad g: \R^n \to \R^R\]
\[\quad \text{ then }\lim\limits_{x \to a}(g\circ f)(x) = l\]

\begin{defn}
  1)Let $f: \R^n \to \R \quad and \quad\phi \neq V \subset Dom(f)$

We say that $f$ is continous on V if:
\[\forall_(x_n) \subset V (x_n) \xrightarrow{C.V} x \in dom(f) \implies \lim\limits_{n \to +\infty}f(x_n) = f(x)\]
2) $f = (f_1,\dots,f_2): \R^n \to \R^n$ continous on $V \subset Dom(f) = \displaystyle\bigcap_{j=i}^n Dom(f_j)$
$\Leftrightarrow \forall_j \in \{1,2,\dots,m\}f_j : \R^n \to \R$ is continous on V
\end{defn}

\section{\underline{The case of $f: \R^2 \to \R$}}

\underline{\textbf{Euclidian distance}}
\[
\begin{aligned}
  d_1((x_1,x_2),(y_1,y_2)) &= |y_1 - x_1|+|y_2 -x_2| \\
d_2((x_1,x_2),(y_1,y_2)) &= \sqrt{(y_1 - x_1)^2+(y_2 -x_2)^2} \\
  d_{\infty}((x_1,x_2),(y_1,y_2)) &= Sup(|y_1 - x_1|,|y_2 -x_2|) \\
\end{aligned}
\]

\[\{(x,y) \in \R^2: d((x,y),(0,0))<1\}\]

$d= d1$ (graph that represents $\{(x,y) \in \R^2: d((x,y),(0,0))<1\}$ expression using d1 distance def)

$d= d2$(graph that represents $\{(x,y) \in \R^2: d((x,y),(0,0))<1\}$ expression using d2 distance def)


$d= d_{\infty}$(graph that represents $\{(x,y) \in \R^2: d((x,y),(0,0))<1\}$ expression using d infty distance def)


\underline{\textbf{limit:}}

$\lim\limits_{(x,y) \to (a,b)}f(x,y) =l \quad / l \in \R$
\[\Leftrightarrow \forall \epsilon>0, \exists \delta>0; 0<d((x,y),(a,b))<\delta \implies |f(x,y) -l| < \epsilon\]

\underline{\textbf{Example:}}

\[f(x,y) = \frac{x^3}{x^2 + y^2}\]
$\lim\limits_{(x,y)\to(0,0)}f(x,y) = 0$
\[
\begin{aligned}
  |f(x,y)| = |\frac{x^3}{x^2 + y^2}| = |x|\frac{x^2}{x^2 + y^2} &\le|x|\\
                                                                &\le|x|+|y|\\
                                                                &d_1((x,y),(0,0)) \\
                                                                \Leftrightarrow \forall \epsilon>0, \exists \delta=\epsilon>0,\forall(x,y) \in \R^2 -\{(0,0)\},\quad 0< |x|+|y| <\delta \implies f(x,y)  < \epsilon
\end{aligned}
\]

\underline{\textbf{Using polar coordinates:}}
\[Let: \quad \begin{cases} 
x = a + rcos(\theta) \quad ; \theta \in \R \\
y = b = rsin(\theta) \quad r> 0
\end{cases}\]

\[
\begin{aligned}
  d_2 = ((x,y), (a,b)) &= \sqrt{(x - a)^2 + (y - b)^2} \\
                       &= \sqrt{(rcos(\theta))^2 +(rsin(\theta))^2} \\
                       &= \sqrt{r^2} = r
\end{aligned}
\]

\[
  l = \in \overline{\R} = \R \cup \{-\infty , \infty\}
\]
\[
\begin{aligned}
  \lim\limits_{(x,y) \to (a,b)} f(x,y) = l &\Leftrightarrow \forall \epsilon>0, \exists \delta>0, 0<r <\delta\\
                         &\implies|f(a + rcos(\theta), b+ rsin(\theta)) - l| < \epsilon \\
                         &\Leftrightarrow\forall \theta \in \R; \lim\limits_{r \xrightarrow{>} 0} f(a + rcos(\theta), b+ rsin(\theta)) =l
\end{aligned}
\]

\underline{\textbf{Example:}}
\[f(x,y) = \frac{x^2 - y^2}{x^2 + y^2}\]
\[(a,b) = (0,0)\]
\[
\begin{aligned}
  f(a + rcos(\theta), b + rsin(\theta)) &= \frac{r^2cos^2(\theta) - r^2sin^2(\theta)}{r^2cos^2(\theta) + r^2sin^2(\theta)} \\
                                        &= cos^2(\theta) - sin^2(\theta)
\end{aligned}
\]
$\theta = 0, l =1$

$\theta = \frac{\pi}{2}, l =-1$

$\implies$ limit doesn't exist 


\underline{\textbf{Example02:}}
\[f(x,y) = \frac{x^3 + y^3}{x^2 + y^2}\]
\[
\begin{aligned}
 f(rcos(\theta), rsin(\theta)) &= \frac{r^3(cos^3(\theta)+ sin^3(\theta))}{r^2} \\
                                &=r(cos^3(\theta)+ sin^3(\theta))
\end{aligned}
\]

\[
0 \le | f(rcos(\theta), rsin(\theta))| = r
\]
\[
  |cos^3(\theta) + sin^3(\theta)| \le 2r \to 0
\]

So $\lim\limits_{(x,y)\to(0,0)}\dfrac{x^3 + y^3}{x^2 + y^2} = 0$




\section{\underline{The case of $f: \R^3 \to \R$}}

$\lim\limits_{(x,y,z) \to (x_0,y_0,z_0)} f(x,y,z) = l$

\underline{\textbf{Using cylindrical coordinates:}}
\[
\begin{cases}
x = x_0 + rcos(\theta) \quad ;r >0\\
y = y_0 + rsin(\theta) \quad \theta \in \R \\
z = z
\end{cases}
\]

\[
\begin{aligned}
  d_2((x,y,z),(x_0,y_0,z_0)) &= \sqrt{(x - x_0)^2 + (y - y_0)^2 + (z -z_0)^2} \\
                             &= \sqrt{r + (z - z_0)^2}
\end{aligned}
\]

We note $g_{\theta} = f(x_0 + rcos(\theta),y_0 + rsin(\theta),z)$
\[\forall \epsilon >0; \exists \delta > 0; d_2((r,z),(0,z_0)) < \delta \implies |g_{\theta}(r,z) - l| < \epsilon\]
\[
\begin{aligned}
  \lim\limits_{(x,y,z)\to(x_0,y_0,z_0)} f(x,y,z) &= l \\
                                                 &\Leftrightarrow \forall\theta \in \R,\lim\limits_{(r,z) \to (0,z_0)}g_{\theta}(r,z) = l
\end{aligned}
\]

\underline{\textbf{Example:}}
\[f(x,y,z) = \frac{x^2 + y^2 - z^2}{x^2 + y^2 + z^2}\]
\[g_{\theta}(r,z) = \frac{r^2 - z^2}{r^2 + z^2}\]

Since $\lim g_{\theta}(r,z)$ does not exist, then
$\lim\limits_{(x,y,z)\to(0,0,0)} \dfrac{x^2 + y^2 - z^2}{x^2 + y^2 + z^2}$ \quad doesn't exist

\subsection{\underline{using spherical coordinates:}}
\[
\begin{cases}
x = x_0 + rcos(\theta)cos(\varphi) \\
y = y_0 + rcos(\theta)sin(\varphi) \\
z = z_0 + rsin(\theta)
\end{cases}
\]
\[g_{\theta,r}(r) = f(x,y,z)\]
\[d_2((x,y,z),(x_0,y_0,z_0)) = r\]
(graph of a line in space and its projection on the (x,y,0) plane varphi and theta and )

So; 
\[
\begin{aligned}
  \lim\limits_{(x,y,z) \to (x_0, y_0, z_0)} f(x,y,z) &= l \\
                                                     &\Leftrightarrow \forall \theta, \varphi,\quad \lim g_{\theta,r}(r) =l
\end{aligned}
\]

\underline{\textbf{Example:}}
\[
\begin{aligned}
  f(x,y,z) &= \frac{x^2 + y^2 -z^2}{x^2 + y^2 + z^2} \\
\end{aligned}
\]
************************************************** 

\section{partials derivatives:}

the process of "partial differentiation" consists of deriving a function of several variables with respect to one of its several independent variables with respect to one of its several indepedent variables \par
the result referred to as the "partal derivative of $f$ with respect to the chosen independent variable.

\begin{defn}
  let  $f : \mathbb{R}^2 \to \mathbb{R}; \quad (a,b) \in dom(f)$

  1) if the limit $\lim\limits_{h\to 0} \frac{f(a+h,b)- f(a,b)}{h} \in \mathbb{R}$ 
  we call them the first partial derivable of $f$ with respect the first independent variable $x$, noted by :
  $\frac{\partial f}{partial x}(a,b)$
  
  2) if the limit $\lim\limits_{h\to 0} \frac{f(a+h,b)- f(a,b)}{h} \in \mathbb{R}$,  we call them
  the first partial derivative of $f$ at (a,b) with respect the second independant variable, denoted by :
  $\frac{\partial f}{\partial y}(a,b)$

  3) we defined two function, \quad $f_x = \frac{\partial f}{\partial x} \text{ and }  \frac{\partial f}{\partial y}$ by: 

    \[
      f_x= \frac{\partial f}{\partial x} :  \mathbb{R}^2 \to \mathbb{R}
      \]
    \[
  (x,y) \mapsto  \frac{\partial f}{\partial x} (x,y) = \lim\limits_{h\to 0} \frac{f(x+h,y)- f(x,y)}{h}
      \]


\end{defn}

\subsection*{\underline{example:}}
1) $f(x,y) = \phi(x)\psi(x)$
\[
\begin{aligned}
  \frac{\partial f}{\partial x}(x,y) : \mathbb{R}^2
\end{aligned}
\]


************************
\section{Geometric interpretation:}
(graph )
/**
**/

the tangent line of the curve $z = f(x,b)$ at $(a,f(a,b))$ \\
is directed by  \quad $(1, 0, \frac{\partial f}{\partial x}(a,b))$
$$
\begin{cases}
  z = f(a,b) + (x-a)\frac{\partial f}{\partial x}(a,b) \\
  y = b
\end{cases}
$$

$\to$ the tangent line of the cuvrve $z = f(a,y) $ at $(a,b,f(a,b))$
$$
\begin{cases}
  z = f(a,b) + (y-b)\frac{\partial f}{\partial y}(a,b) \\
  y = b
\end{cases}
$$
$\to$which is directed by: \quad$(0,1, \frac{\partial f}{\partial y}(a,b))$


the tangent plane of the graph of $f$ at 
$(a,b,f(a,b))$ , denoted by $T_{((a,b,f(z,b))}Gr_f$
is defined by 
\[
\begin{aligned}
  \{(x,y,z) \in \mathbb{R}^3\ :\ (x,y,z) = (a,b,f(a,b)) + (x-a)(1,0,\frac{\partial f}{\partial x}(a,b)) (y-b)(0,1,\frac{\partial f}{\partial y}(a,b))\} \\
  T_{(a,b,f(a,b))}Gr_f =\{(x,y,z) \in \mathbb{R}^3\ : z = f(a,b) + \frac{\partial f}{\partial x}(a,b)(x-a) +\frac{\partial f}{\partial y}(a,b)(y-b)
\end{aligned}
\]

\begin{defn}
  1) Let
  \[
    f : \mathbb{R}^n \to \mathbb{R}
    \]
  \[
    (x_1,\dots,x_n) \mapsto f(x_1,\dots,x_n)
    \]
$$
\forall i \in {1,2,\dots,n} f_{x_i}= \frac{\partial f}{\partial x_i} : \mathbb{R}^n \to \mathbb{R}
(x_1,\dots,x_n)\mapsto \frac{\partial f}{\partial x_i}(x_1,\dots,x_n)= \lim\limits_{h \to 0}\frac{f(x + he_1)-f(x)}{h}
$$
$\frac{\partial f}{\partial x_i}$ is the i-th first derivative of $f$ or the first partial derivative of 
$f =\lim\limits_{h \to 0}\frac{f(x1,\dots, x_i + h, \dots,x_n)-f(x_1,\dots,x_n)}{h}$
with respect the i-th independent  variable \par
when m>1: \quad  $f=(f_1,\dots,f_m): \mathbb{R}^n \to \mathbb{R}^m$
$\forall_{i \in \{1,\dots,n\}}; \frac{\partial f}{\partial x_i}= (\frac{\partial f_1}{\partial x_i},\dots,\frac{\partial f_n}{\partial x_i}) :\mathbb{R}^n \to \mathbb{R}^m$ \par
the tangent plan of :
$ Gr_f = \{(x,y) \in \mathbb{R}^n\mathbb{R}^m : y =f(x)\}$ at $(a,f(a))$ / $a \in \mathbb{R}^n$ , denoted by 
$T_{(a,f(a)} Gr_f$ is :
\[
  T_{(a,f(a)}Gr_f = \{(x,y) \in \mathbb{R}^n \mathbb{R}^m: y = f(a) + \sum_{i = 1}^n(x_i - a_i)\frac{\partial f}{\partial x_i}(a)\} \subset \mathbb{R}^n \times \mathbb{R}^m = R^{n + m}
\]
\end{defn}
\subsection{\underline{Example:1}}
\[
  f: \mathbb{R}^3 \to \mathbb{R}
\]
\[
  (x,y,z) \mapsto xy +yz^2 + xy
\]
\begin{defn}{Second "order" partial derivatives\par}
  1) Let  $f : \mathbb{R}^2 \to \mathbb{R}$ and $\dfrac{\partial f}{\partial x} , \dfrac{\partial f}{\partial y}$ its first  partial deriviative :
  \[
  \begin{aligned}
    f_{xx} = \frac{\partial^2 f}{\partial x^2} = \frac{\partial}{\partial x}(\frac{\partial f}{\partial x})(x,y) = \lim\limits_{h \to 0}\frac{\frac{\partial f}{\partial x}(x+h,y)-\frac{\partial f}{\partial x}(x,y)}{h} \\
    f_{xy} = \frac{\partial^2 f}{\partial y\partial x} = \frac{\partial y}{\partial y}(\frac{\partial f}{\partial x})(x,y) = \lim\limits_{h \to 0}\frac{\frac{\partial f}{\partial x}(x,y + h )-\frac{\partial f}{\partial x}(x,y)}{h} \\
    f_{yx} = \frac{\partial^2 f}{\partial y \partial x} = \frac{\partial}{\partial x}(\frac{\partial f}{\partial y})(x,y) = \lim\limits_{h \to 0}\frac{\frac{\partial f}{\partial y}(x+h,y)-\frac{\partial f}{\partial y}(x,y)}{h} \\
    f_{yy} = \frac{\partial^2 f}{\partial y^2} = \frac{\partial}{\partial y}(\frac{\partial f}{\partial y})(x,y) = \lim\limits_{h \to 0}\frac{\frac{\partial f}{\partial y}(x,y+h)-\frac{\partial f}{\partial y}(x,y)}{h} \\
  \end{aligned}
  \]
  2) Let $f: \mathbb{R}^n \to \mathbb{R}$ and 
  $\frac{\partial f}{\partial x_i}; i \in \{1,\dots,n\}$ its first partial \par
  $$\forall{i,j}\in \{1,\dots,n\}\quad \frac{\partial^2 f}{\partial x_i\partial x_j }(x_1,\dots,x_n) \mapsto
  \lim\limits_{h \to 0}\frac{\frac{\partial f}{\partial x_i}(x_1,\dots,x_i+h,\dots,x_n)-\frac{\partial f}{\partial x_j}(x_1,\dots,x_j+h,\dots,x_n)}{h}
  $$

  when $i =j\text{ } \frac{\partial^2 f}{\partial x_i \partial x_j} =\frac{\partial^2 f}{\partial x_i \partial x_j}$ is called the second partial derivative of $f$ with respect  $x_i$


  when $i \neq j\text{ } \frac{\partial^2 f}{\partial x_i \partial x_j} $ is called the mixed second partial derivative of $f$ with respect  $x_i$ \par

there is  $n^2$ second partial derivatives of $f$ \par

3) If $f = (f_1,\dots,f_n) : \mathbb{R}^n \to \mathbb{R}^m$
\[
  \forall_{i,j} \in \{1,\dots,n\}; \frac{\partial^2 f}{\partial x_i \partial x_j} = (\frac{\partial^2 f_1}{\partial x_i \partial x_j},\dots,\frac{\partial^2 f_m}{\partial x_i \partial x_j})
\]
\end{defn}

\begin{defn}{Shwarz's theorem\par}

  Let $f: \mathbb{R}^n \to \mathbb{R}^m$ \par
  If $\frac{\partial^2 f}{\partial x_i \partial x_j}$ and $\frac{\partial^2 f}{\partial x_j \partial x_i}$ continous on open subset $D \subset \mathbb{R}^n$ then 
 $\frac{\partial^2 f}{\partial x_i \partial x_j}$ = $\frac{\partial^2 f}{\partial x_j \partial x_i}$

\end{defn}

\begin{defn}
  \underline{Higher order parital derivatives}
  Let $p \in \mathbb{N}^*; f : \mathbb{R}^n \to \mathbb{R}^m$
  \[
  \frac{\partial f}{\partial x_1^{P_1}\partial x_2^{P_2}\dots\partial x_n^{P_n}}, \quad \forall P_1,\dots,P_n \in \mathbb{N} \text{\quad such that } P_1 + P_2 + \dots + P_m = P
  \]

\end{defn}
  \subsection{\underline{Example:}}
  \[
    f : \mathbb{R}^n \to \mathbb{R}
  \]
  \[\frac{\partial^3 f}{\partial x^p \partial y^q}\quad/ p+q = 3 \quad \]

\section{Differentials of a function}

If 
\[
f: \mathbb{R} \to \mathbb{R}, \quad x \mapsto f(x)
\] 
and $f'$ is its derivative function,  
then the differential of $f$, denoted $df$, is the function
\[
df: \mathbb{R} \to L(\mathbb{R},\mathbb{R}) 
= \{\phi : \mathbb{R} \to \mathbb{R} \mid \phi \text{ linear} \},
\]
\[
x \mapsto d_x f: \mathbb{R} \to \mathbb{R}, \quad h \mapsto f'(x) \cdot h
\]

\begin{defn}
Let $f : \mathbb{R}^n \to \mathbb{R}^m$ be a function such that all first partial derivatives exist.  
Then the differential of $f$ is the function
\[
\begin{aligned}
df : \mathbb{R}^n &\to L(\mathbb{R}^n, \mathbb{R}^m),\\
x &\mapsto d_x f : \mathbb{R}^n \to \mathbb{R}^m, \quad 
h = (h_1,\dots,h_n) \mapsto \sum_{i=1}^n h_i \frac{\partial f}{\partial x_i}(x)
\end{aligned}
\]
\end{defn}
\underline{\textbf{Remark:}}
If 
\[
 \begin{aligned}
  f: \mathbb{R}^n \to \mathbb{R}^m \text{ is linear, then :} 
  \forall_{x \in } \mathbb{R}^n: \quad d_x f = f
\end{aligned}
\]

\underline{\textbf{In Particular: }} $f = pr_i\quad (x_1,\dots,x_n) \mapsto x_i$
\[
\begin{aligned}
  \forall_x , d_xf =f=pr_i =dx_i
  \quad \text{so: }: pr_i = dx_i :
  \mathbb{R}^n \to \mathbb{R}\\
  (h_1, \dots, h_n) \mapsto h \\
  \text{if m = 1:} 
  d_xf = \sum_{i = 1}^n \frac{\partial f}{\partial x_i}(x)\text{ } dx_i
\end{aligned}
\]

\subsection{\underline{Example:}}
\[
\begin{aligned}
  f: \mathbb{R}^2 \to \mathbb{R}
  (x,y) \mapsto f(x,y) \\
  \begin{cases}
    d_{(x,y)}f = \frac{\partial f}{\partial x}(x,y)\text{ } +  \frac{\partial f}{\partial y}(x,y) \text{ } dy \\
    df =  \frac{\partial f}{\partial x}dx  \frac{\partial f}{\partial y} dy
  \end{cases}
\end{aligned}
\]
$f(x,y) = xy$; \quad $d_(x,y)f = ydx = xdy$ \par
application  $w = f(x_1,\dots,x_n)$;\quad $\Delta w = f (x_1 + \Delta x_1,\dots,x_n + \Delta x_n)$ \\

then : 
\[
  \Delta w = \sum_{i =1}^n \frac{\partial f}{\partial x_i}(x_1,\dots,x_n)\Delta x_i \to f(x_1,\dots,x_n)
\]
\begin{defn}
\underline{\textbf{Differentialbility}}
Let $f : \mathbb{R}^n \to \mathbb{R}^m$, and $a \in  \mathbb{R}^n $\par
we say that f is differentiable at a if there exist a linear map such that
\[
 L_a:  \mathbb{R}^n \to \mathbb{R}^m \\
 h \mapsto L_a(h) 
\]
\[
 h \mapsto L_a(h) 
\]

such that : 
\[
  \lim\limits_{h \to 0_{\mathbb{R}^n}} \frac{1}{\sqrt{h_1^2 +\dots+h_n^2}} (f(a+h) - f(a) -L_a(h))
\]
(example for n =1 , m=1)
\end{defn}
\underline{\textbf{Remark:}}
If $f$ is differentiable at $(a_1,\dots, a_n)$, then : \par 
\[
\begin{aligned}
  \forall_i \in {1, \dots,n} \text{if we consider } h = (0,\dots,0,t,0,\dots,0) = te_i \\
\lim\limits_{t \to 0} \frac{1}{|t|} (f(a+te_i) - f(a) -L_a(te_i)) \\ 
  \Leftrightarrow \lim\limits_{t \to 0} \frac{1}{t} (f(a+te_i) - f(a) -L_a(te_i)) = L_a(e_i)\\
  \Leftrightarrow \lim\limits_{t \to 0} \frac{1}{t} (f(a_1,a_2,\dots+(a_i + t),\dots,+a_n) - f(a_1,\dots,a_n)   \\
  \frac{\partial f}{\partial x_i}(a) = L_a(e_i) \quad \forall_i \in \{1,\dots,n\}
  L_a(h) &= L_a(\sum_{i = 1}^n h_i e_i) = \sum_{i=1}^n h_iL_a(e_i)\\
         &=\sum_{i = 1}^n h_i\text{ ; } \frac{\partial f}{\partial x_i}= d_af(h) \\
         So\text{, }L_a = d_af
\end{aligned}
\]
so, $f$ is differentiable at  $a$  iff all its first partial derivatives exist and 
\[
  \forall_h \in \mathbb{R}^n : f(a+h) = f(a) + d_af(h) + o(h)
\]
where \quad $\lim\limits_{h \to 0_{\mathbb{R}^n} }\dfrac{1}{\sqrt{h_1^2+\dots+h_n^2}}o(h) = 0 $


\end{document}
