\documentclass{article}
\usepackage[utf8]{inputenc}
\usepackage{amsmath}

\title{lecture one}
\author{Chemistry}
\date {30/09/2025}

\begin{document}
\maketitle


\section*{chapter i: solution chemistry}
\subsection{General information Free energy (G) and equilibrium constant (k) }

\subsubsection{spontaneity criteria: "second law of thermodynamic"}

$$\Delta S = \Delta S_{tot} = \Delta S_{sys} + \Delta S_{exch} / \Delta S_{exch} = - \Delta S_{surr}  
where \Delta
$$

$\Delta S_{tot} >= 0 \to $   
$\Delta S_{tot} = 0 \to  $    reversable (ideal) \\
$\Delta S_{tot} > 0 \to $    irreversable (real)

$\Delta S_{tot} \to process is impossible (non_spontanous)$


However , in practice, the calculation of $\delta S_{exch}$ is not always possible $\to$ limitation

$\to$ We introduce a newconcept : Free energy or Gibbs energy  \\
with  
$$
G = H - TS \quad,\quad  dG = dH - TdS 
$$

$$
\Delta G = \Delta H  - T \Delta S
$$

where $\Delta H = enthalpy $ \\ $ \Delta S : entropy $ 

G is very used for open systems (i.e chemical reaction)  \\
G is a state function 

then, the spontanuty criteria are :

$\Delta G < 0 \quad , $  the porocess is spontanous, G decreases
$\Delta G > 0 \quad , $  the porocess is non-spontanous, it could not take place 
$\Delta G = 0 \quad , $ (G = const),  we are at equilibrium in the studied conditions 

\underline {Example} : Chemical reaction  :

$ A + B \to C + D$
$\Delta G < 0 \ to $ the forward reaction is spontanous  $\quad\to: forward direction$  \\
$\Delta G > 0 \ to $ the reverse reaction is spontanous  $\quad\leftarrow: reverse direction$ 

\subsection{Calculation of free energy:}
$\Delta G^0 = \Delta H^0 - T \Delta S^0$

$\ \Delta H^0 \Delta S^0 (S^0) \to$ thermodynaamic data 

\underline{Example:} The vaporization of water at room temperature : 

$H_2O(l) \to^{\Delta G = ?} H_2O (g) $

Data: 
% \begin{tabular}
%
% \end{tabular}

\begin{align*}
\Delta G^\circ &= \Delta H^\circ - T\Delta S^\circ \\[6pt]
\Delta H^\circ &= \Delta H^\circ_\mathrm{f}(H_2\mathrm{(g)}) - \Delta H^\circ_\mathrm{f}(H_2\mathrm{(l)}) \\
               &= -241.82 - (-285.083) = 44.01 \; \text{kJ mol}^{-1} \\[6pt]
\Delta S^\circ &= S^\circ_{298\,\mathrm{H_2O}(g)} - S^\circ_{298\,\mathrm{H_2O}(l)} \\
               &= 118.8 \;\text{J mol}^{-1}\text{K}^{-1}
\end{align*}

\subsection{Free energy for chemical reaction}
Let's have the following reaction: $aA + bB \xrightarrow{\Delta g_R} cC + dD$

$\Delta G_R^\circ  = \sum n, \Delta G_f^\circ \text{(Products)} - \sum n, \Delta G_f^\circ; (reagents)$  / F = formation \\
The Hess
s law 

\subsection{Chemcial equilibrium: }
The chemical reactions are not always \underline {complete}, they are not always accompanied by total desparition of reagents. 

Too many reactions are \underline {partial},  reagents and products are present simultanously 

During equilibrium, the rate of forward reaction equals the rate of revers. reaction .

aA + bB $\to$ cC + dD / (kf up arrow and kr / f : forward/ r :reverse)

Reagents and products are present at the same proportion $\rightarrow$ the mass action law 

\subsection{The massaction law: } 
This law is represented by a constant((f(t))  called equilibrium \underline constant = $K_eq$

with: $ K_eq = \frac{[C]^c[D]^d}{[A]^a[B]^b}$

It was proved that 
$\Delta G(i) = \Delta G^\circ + RT\ln{K}$

$\to \text{equilibruim} \to \Delta G = 0 \to \Delta G^\circ = -RT\ln{K}$




\end{document}

