\documentclass[12pt]{article}
\usepackage[utf8]{inputenc}
\usepackage[T1]{fontenc}
\usepackage{amsmath, amssymb, amsthm}
\usepackage{siunitx}
\usepackage{tikz}
\usepackage{pgfplots}
\usepackage{geometry}
\usepackage{titlesec}
\usepackage{graphicx}

\geometry{a4paper, margin=1in}
\pgfplotsset{compat=1.18}
\usetikzlibrary{3d, arrows.meta}

% Theorem environments
\newtheorem{definition}{Definition}
\newtheorem{proposition}{Proposition}
\newtheorem{remark}{Remark}
\newtheorem{example}{Example}
\newtheorem{theorem}{Theorem}

% Paragraph formatting
\setlength{\parindent}{1.5em}
\setlength{\parskip}{1em}

% Add extra vertical space
\titleformat{\section}{\normalfont\Large\bfseries}{\thesection}{1em}{}
\titlespacing*{\section}{0pt}{2ex plus 1ex minus .2ex}{1ex plus .2ex}

\title{Chapter 01: Electrochemical Kinetics}
\author{Notes from Prof Aoudj}

\hbadness=10000
\hfuzz=\maxdimen

\newcommand{\N}{\mathbb{N}}
\newcommand{\R}{\mathbb{R}}
\newcommand{\Z}{\mathbb{Z}}
\newcommand{\Q}{\mathbb{Q}}
\newcommand{\C}{\mathbb{C}}
\newcommand{\E}{\mathbb{E}}
\newcommand{\F}{\mathbb{F}}

\newcommand{\pd}[2]{\dfrac{\partial #1}{\partial #2}}
\newcommand{\pdd}[3]{\dfrac{\partial^2 #1}{\partial #2\,\partial #3}}
\newcommand{\pddx}[2]{\dfrac{\partial^2 #1}{\partial #2^2}}
\newcommand{\dd}[2]{\dfrac{\mathrm{d} #1}{\mathrm{d} #2}}
\newcommand{\ddd}[2]{\dfrac{\mathrm{d}^2 #1}{\mathrm{d} #2^2}}
\newcommand{\diff}{\,\mathrm{d}}

\newcommand{\vect}[1]{\begin{pmatrix}#1\end{pmatrix}}

\begin{document}

\maketitle

\vspace{3em}

\section{Introduction}

The thermodynamic study of an electrochemical system is insufficient to study the principal 
phenomena (it only tells us whether a transformation is possible or spontaneous).
The study of the reaction rate provides complementary elements.


\section{Overpotential and polarization curve}
\subsection{Overpotential}

If an electrode is crossed by a current I,(because of a polarization:put a potential).It takes a potential E 
$\boxed{E \neq E_{eq}}$. So, the overpotential $\boxed{\mu : \mu = E - E_{eq}}$

\underline{\textbf{Reminder:}}

what's $E_{eq}$, Eq = Equilibrium potential given by Nernst formula:
For Redox reaction: 
$$ \quad \quad Ox + ne^- \rightleftharpoons red \to E_{eq_{Ox/Red}} = E^{\circ}_{Ox/Red}+
\frac{0.06}{n} log(\frac{C_{Ox}}{C_{Red}})$$

$E_eq$ corresponds to the potential when I = 0 

The electrochemcial reaction takes place in the direction where $\Delta G < 0$ , So we 
have always $\boxed{\mu.I \geq 0 }$

By converntion, $I = I^+$ (Positive) for oxidation 

 $I = I^-$ (negative) for reduction
 
 Let's see (I = f(E)) , $E(v) \neq U $ (E(v) is potential, U is ddp (difference de potential)
 
\begin{center}
\includegraphics[width=0.8\textwidth]{photos/I_function_of_E.jpg}
\end{center}

\end{document}
